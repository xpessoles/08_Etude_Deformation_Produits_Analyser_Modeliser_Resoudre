\documentclass[10pt,fleqn]{article} % Default font size and left-justified equations
\usepackage[%
    pdftitle={Résistance des matériaux : Modélisation des pièces déformables - Flexion},
    pdfauthor={Xavier Pessoles}]{hyperref}

\input{style/new_style}
\input{style/macros_SII}

\fichetrue
%\fichefalse

\proftrue
%\proffalse

%\tdtrue
\tdfalse

%\courstrue
\coursfalse

% -------------------------------------
% Déclaration des titres
% -------------------------------------

\def\discipline{Sciences \\Industrielles de \\ l'Ingénieur}
\def\xxtete{Sciences Industrielles de l'Ingénieur}

\def\classe{Fiche PT}
\def\xxnumpartie{Cycle RdM}
\def\xxpartie{Résistance des matériaux}

\def\xxnumchapitre{Chapitre 4 \vspace{.2cm}}
\def\xxchapitre{\hspace{.12cm} Modélisation des pièces déformables en flexion}

\def\xxposongletx{2}
\def\xxposonglettext{1.45}
\def\xxposonglety{19}%16

\def\xxonglet{Cycle RdM -- Ch. 4}

\def\xxactivite{Fiche}
\def\xxauteur{\textsl{Xavier Pessoles}}

\def\xxcompetences{%
\textsl{%
\textbf{Savoirs et compétences :}\\
}}

\def\xxfigures{
%incgraphics[width=.8\textwidth]{}%images/prot_01}
}%figues de la page de garde

\def\xxpied{%
Cycle RdM -- Modélisation des pièces déformables\\
Flexion -- \xxactivite%
}

\setcounter{secnumdepth}{5}
%---------------------------------------------------------------------------


\begin{document}
%\chapterimage{png/Fond_Cin}
\input{style/new_pagegarde}
\vspace{2cm}
\pagestyle{fancy}
\thispagestyle{plain}

\section{Définitions}
\begin{defi} -- \textbf{Torseur des sollicitations}  ~\\

\begin{minipage}[c]{.65\linewidth}
Pour une sollicitation en torsion, le torseur de cohésion est de la forme : 
$$
\{\mathcal{T}_{\text{coh}}\} = \torseurcol{0}{T_y}{0}{0}{0}{M_{fz}}{G,\left(\vx{}, \vy{}, \vz{} \right)} 
\text{ ou } 
\{\mathcal{T}_{\text{coh}}\} = \torseurcol{0}{0}{T_z}{0}{M_{fy}}{0}{G,\left(\vx{}, \vy{}, \vz{} \right)} 
.$$
\end{minipage} \hfill
\begin{minipage}[c]{.3\linewidth}
\begin{center}
\includegraphics[width=\linewidth]{images/flexion}
\end{center}
\end{minipage}

\end{defi}

\begin{defi} -- 
\textbf{Contrainte et déformations} ~\\
\begin{minipage}[c]{.7\linewidth}
En flexion, les contraintes tangentielles étant négligeables devant les contraintes normales, on a : 
$$ \sigma = - \dfrac{M_{fz}}{I_{Gz}}y
\quad 
\text{avec}
\begin{array}{ll}
\sigma &\text{contrainte en \text{MPa},}\\
M_{fz} &\text{moment de flexion autour de $\vz{}$ en Nmm,} \\
I_{Gz} &\text{moment quadratique par rapport à l'axe Gx en mm}^4 \text{ et} \\
y &\text{distance à la fibre neutre en mm.}
\end{array}
$$
La déformée $y(x)$ de la poutre vérifie l'équation différentielle suivante :

$$
EI_{Gz}\dfrac{\text{d}^2 y(x) }{\text{d} x^2} = M_{fz}
\quad 
\text{avec}
\begin{array}{ll}
E &\text{module de Young en MPa,}\\
I_{Gz} &\text{moment quadratique par rapport à l'axe} \left(G,\vect{x} \right) \text{en mm}^4, \\
y(x) &\text{déformée de la poutre en mm,} \\
M_{fz} &\text{moment fléchissant en Nm.} \\
\end{array}
$$

\end{minipage} \hfill
\begin{minipage}[c]{.25\linewidth}
\begin{center}
\includegraphics[width=.9\linewidth]{images/contrainte_flexion}
\end{center}
\end{minipage}
\end{defi}


\begin{prop}
\textbf{Propriétés des sections droites}
On considère une poutre de section droite $S$ et d'axe $\left(G,\vect{x} \right)$. On note :
\begin{itemize}
\item moment quadratique de $S$ par rapport à $\left(G,\vect{y} \right)$ : $I_{Gy} = \iint\limits_S z^2 \text{d}S$;
\item moment quadratique de $S$ par rapport à $\left(G,\vect{z} \right)$ : $I_{Gz} = \iint\limits_S y^2 \text{d}S$;
\item moment polaire de $S$ par rapport à $\left(G,\vect{x} \right)$ : $I_{Gx} = \iint\limits_S \left( y^2 + z^2 \right) \text{d}S$;
\item $I_{Gx}=I_{Gy}+I_{Gz}$.
\end{itemize}
\end{prop}

\begin{theorem}
\textbf{Théorème de Huygens} ~\\
On a, avec $\vect{AG}=\left(a,b,c \right)$ :
$$
I_{Ay}=I_{Gy} + Sc^2 
\quad 
I_{Az}=I_{Gz} + Sb^2 
\quad 
I_{Ax}=I_{Gx} + S\left(b^2+c^2 \right). 
$$
\end{theorem}

\subsection*{Cas du cisaillement}

\begin{defi} -- 
\textbf{Contrainte et déformations} ~\\

Pour une section cisaillée, la contrainte tangentielle est de la forme : 
$$
\tau = \dfrac{T_y}{S} \quad \text{avec } S \text{ section cisaillée.}
$$
\end{defi}

\newpage

\begin{resultat}
\textbf{Dimensionnement d'une pièce cisaillée} ~\\
On a : $|\tau| < \dfrac{Rg}{s}$ avec $s$ coefficient de sécurité et $Rg$ limite au glissement tel que $Rg = \xi Re$ :
\begin{itemize}
\item acier doux (\% C < 0,2) : $\xi =0,5$;
\item acier mi-doux (\% C compris entre 0,2 et 0,32) : $\xi =0,6$;
\item acier mi-durs (\% C compris entre 0,32 et 0,45) : $\xi =0,7$;
\item acier durs (\% C>0,45) : $\xi =0,8$;
\item fontes (\% C>1,7) : $\xi \in [0,77;1]$.
\end{itemize}

\end{resultat}
\end{document}


