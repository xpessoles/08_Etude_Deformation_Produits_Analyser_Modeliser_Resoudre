\documentclass[10pt,fleqn]{article} % Default font size and left-justified equations
\usepackage[%
    pdftitle={Résistance des matériaux : Modélisation des pièces déformables - Flexion},
    pdfauthor={Xavier Pessoles}]{hyperref}

\input{style/new_style}
\input{style/macros_SII}

\fichetrue
%\fichefalse

\proftrue
%\proffalse

%\tdtrue
\tdfalse

%\courstrue
\coursfalse

% -------------------------------------
% Déclaration des titres
% -------------------------------------

\def\discipline{Sciences \\Industrielles de \\ l'Ingénieur}
\def\xxtete{Sciences Industrielles de l'Ingénieur}

\def\classe{Fiche PT}
\def\xxnumpartie{Cycle RdM}
\def\xxpartie{Résistance des matériaux}

\def\xxnumchapitre{Chapitre 3}
\def\xxchapitre{\hspace{.12cm} Modélisation des pièces déformables en flexion}

\def\xxposongletx{2}
\def\xxposonglettext{1.45}
\def\xxposonglety{19}%16

\def\xxonglet{Cycle RdM -- Ch. 3}

\def\xxactivite{Fiche}
\def\xxauteur{\textsl{Xavier Pessoles}}

\def\xxcompetences{%
\textsl{%
\textbf{Savoirs et compétences :}\\
}}

\def\xxfigures{
%incgraphics[width=.8\textwidth]{}%images/prot_01}
}%figues de la page de garde

\def\xxpied{%
Cycle RdM -- Modélisation des pièces déformables\\
Flexion -- \xxactivite%
}

\setcounter{secnumdepth}{5}
%---------------------------------------------------------------------------


\begin{document}
%\chapterimage{png/Fond_Cin}
\input{style/new_pagegarde}
\vspace{2cm}
\pagestyle{fancy}
\thispagestyle{plain}

\section{Définitions}
\begin{defi} -- \textbf{Torseur des sollicitations}  ~\\

\begin{minipage}[c]{.65\linewidth}
Pour une sollicitation en torsion, le torseur de cohésion est de la forme : 
$$
\{\mathcal{T}_{\text{coh}}\} = \torseurcol{0}{T_y}{0}{0}{0}{M_{fz}}{G,\left(\vx{}, \vy{}, \vz{} \right)} 
\text{ ou } 
\{\mathcal{T}_{\text{coh}}\} = \torseurcol{0}{0}{T_z}{0}{M_{fy}}{0}{G,\left(\vx{}, \vy{}, \vz{} \right)} 
.$$
\end{minipage} \hfill
\begin{minipage}[c]{.3\linewidth}
\begin{center}
\includegraphics[width=\linewidth]{images/flexion}
\end{center}
\end{minipage}

\end{defi}

\begin{defi} -- 
\textbf{Contrainte et déformations} ~\\
\begin{minipage}[c]{.7\linewidth}
En flexion, les contraintes tangentielles étant négligeables devant les contraintes normales, on a : 
$$ \sigma = - \dfrac{M_{fz}}{I_{Gz}}y
\quad 
\text{avec}
\begin{array}{ll}
\sigma &\text{contrainte en \text{MPa},}\\
M_{fz} &\text{moment de flexion autour de $\vz{}$ en Nmm,} \\
I_{Gz} &\text{moment quadratique par rapport à l'axe Gx en mm}^4 \text{ et} \\
y &\text{distance à la fibre neutre en mm.}
\end{array}
$$
La déformée $y(x)$ de la poutre vérifie l'équation différentielle suivante :

$$
EI_{Gz}\dfrac{\text{d}^2 y(x) }{\text{d} x^2} = M_{fz}
\quad 
\text{avec}
\begin{array}{ll}
E &\text{module de Young en MPa,}\\
I_{Gz} &\text{moment quadratique par rapport à l'axe} \left(G,\vect{x} \right) \text{en mm}^4, \\
y(x) &\text{déformée de la poutre en mm,} \\
M_{fz} &\text{moment fléchissant en Nm.} \\
\end{array}
$$

\end{minipage} \hfill
\begin{minipage}[c]{.25\linewidth}
\begin{center}
\includegraphics[width=.9\linewidth]{images/contrainte_flexion}
\end{center}
\end{minipage}
\end{defi}

%
%\begin{resultat}
%\textbf{Loi de comportement -- Loi de Hooke} ~\\
%Lorsqu'un matériau est sollicité dans un domaine élastique, contrainte et déformation sont liées par la loi de Hooke : 
%$$ \tau = G \gamma
%\quad 
%\text{avec}
%\begin{array}{ll}
%\sigma &\text{contrainte en \text{MPa}},\\
%\varepsilon &\text{déformation sans dimension,}\\
%E &\text{module de Young en MPa (N}\cdot\text{mm}^2\text{).}
%\end{array}
%$$
%\end{resultat}
%
%\begin{resultat} -- \textbf{Déformations transversales} ~\\
%En traction, la déformation la plus importante est suivant la direction de l'effort. Cependant, cet allongement s'accompagne d'un rétrécissement de la section. Le coefficient de Poisson est défini par (on considère que la direction de traction est suivant $\vect{x}$): 
%$$
%\nu = \dfrac{\varepsilon_y}{\varepsilon_x} = \dfrac{\varepsilon_z}{\varepsilon_x} \simeq 0,3.
%$$
%
%\end{resultat}
%
%
%\begin{resultat} -- \textbf{Dimensionnement à la traction} ~\\
%\begin{itemize}
%\item Dimensionnement en contrainte : pour dimensionner une poutre à la traction avec un coefficient de sécurité $s$ (supérieur à 1), la contrainte maximale ne doit pas dépasser la résistance pratique à l'extension $Rpe = \dfrac{Re}{S}$ :
%$$
%\sigma_{max} \leq \dfrac{Re}{s} \quad \text{(Re : limite d'élasticité en MPa)}.
%$$ 
%\item Dimensionnement en déplacement : le déplacement d'un point ne doit pas dépasser un déplacement limite fixé par le cahier des charges.
%
%\item Suivant la géométrie de la poutre (gorges, rayon de raccordement...), des concentrations de contraintes peuvent apparaître. Ainsi la contrainte maximale est pondérée par un coefficient $Kt$ donnée par des abaques : 
%$$
%\sigma_{max} Kt \leq \dfrac{Re}{s}.
%$$ 
%\end{itemize}
%\end{resultat}
%

%\section{Caractérisation des paramètres}
%Le module de Young $E$, la limite élastique $Re$ et la limite à la rupture $Rm$ sont déterminés grâce à l'essai de traction. 
%\begin{center}
%\includegraphics[width=\linewidth]{images/essai_traction}
%\end{center}

\begin{prop}
\textbf{Propriétés des sections droites}
On considère une poutre de section droite $S$ et d'axe $\left(G,\vect{x} \right)$. On note :
\begin{itemize}
\item moment quadratique de $S$ par rapport à $\left(G,\vect{y} \right)$ : $I_{Gy} = \iint\limits_S z^2 \text{d}S$;
\item moment quadratique de $S$ par rapport à $\left(G,\vect{z} \right)$ : $I_{Gz} = \iint\limits_S y^2 \text{d}S$;
\item moment polaire de $S$ par rapport à $\left(G,\vect{x} \right)$ : $I_{Gx} = \iint\limits_S \left( y^2 + z^2 \right) \text{d}S$;
\item $I_{Gx}=I_{Gy}+I_{Gz}$.
\end{itemize}
\end{prop}

\begin{theorem}
\textbf{Théorème de Huygens} ~\\
On a, avec $\vect{AG}=\left(a,b,c \right)$ :
$$
I_{Ay}=I_{Gy} + Sc^2 
\quad 
I_{Az}=I_{Gz} + Sb^2 
\quad 
I_{Ax}=I_{Gx} + S\left(b^2+c^2 \right). 
$$
\end{theorem}

\end{document}


