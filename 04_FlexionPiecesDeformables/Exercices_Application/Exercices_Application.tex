\documentclass[10pt,fleqn]{article} % Default font size and left-justified equations
\usepackage[%
    pdftitle={RdM : Diagrammes de sollicitations},
    pdfauthor={Xavier Pessoles}]{hyperref}
    
\input{style/new_style}
\input{style/macros_SII}

\usepackage{multicol}
\usepackage{style/schemabloc}
\fichetrue
%\fichefalse

\proftrue
%\proffalse

\tdtrue
%\tdfalse

%\courstrue
\coursfalse

\def\discipline{Sciences \\Industrielles de \\ l'Ingénieur}
\def\xxtete{Sciences Industrielles de l'Ingénieur}

\def\classe{PT -- PT$\star$}
\def\xxnumpartie{Partie n}
\def\xxpartie{Méthode de résolution permettant la détermination des déformations et des contraintes des pièces déformables\\
Analyse, Modélisation, Résolution}

\def\xxnumchapitre{Chapitre n}
\def\xxchapitre{Titre Chapitre}

\def\xxtitreexo{Exercices d'application}
\def\xxsourceexo{\hspace{.2cm} D'après notes de cours PT -- Lycée G. Eiffel, Bordeaux.}


\def\xxposongletx{2}
\def\xxposonglettext{1.45}
\def\xxposonglety{20}
\def\xxonglet{Part. 2 -- Ch. 2}

\def\xxactivite{Applications}
\def\xxauteur{\textsl{Xavier Pessoles}}

\def\xxcompetences{%
\textsl{%
\textbf{Savoirs et compétences :}\\
\noindent \textbf{Résoudre :} à partir des modèles retenus :
\begin{itemize}[label=\ding{112},font=\color{ocre}] 
\item ***
\item ***
\end{itemize}
\begin{itemize}[label=\ding{112},font=\color{ocre}] 
\item ***
\end{itemize}
%
%\noindent \textit{Mod2 -- C4.1 :} Représentation par schéma bloc.
}}

\def\xxfigures{
\includegraphics[width=.8\textwidth]{images/rdm}
}%figues de la page de garde

\def\xxpied{%
Partie 2 -- Découverte des SLCI -- Analyse, Modélisation, Résolution \\
Ch. 2 : Modélisation des Systèmes Linéaires Continus Invariants -- Transformée de Laplace -- \xxactivite%
}


\setcounter{secnumdepth}{5}
%---------------------------------------------------------------------------


\begin{document}
%\chapterimage{png/Fond_Cin}
\input{style/new_pagegarde}
\vspace{10cm}
\pagestyle{fancy}
\thispagestyle{plain}


\def\columnseprulecolor{\color{ocre}}
\setlength{\columnseprule}{0.4pt} 

%\begin{multicols}{2}

\section*{Exercice 1 -- Calcul d'un moment quadratique}
\setcounter{subparagraph}{0}
On donne sur le schéma ci-dessous.
\begin{center}
\includegraphics[width=.45\textwidth]{images/fig_01}
\end{center}

\subparagraph{}
\textit{Déterminer $I_{G_\Sigma,\vect{z}}(x)$.}
\ifprof
\begin{corrige}
$I_{G_\Sigma,\vect{z}}(x) = I_{Gz} = \iint y^2 \text{d}S$

En coordonnées cartésiennes l'élément infinitésimal de surface se note $\text{d}S = \text{d}y \text{d}z$.


\begin{eqnarray*}
I_{Gz} &=& \iint y^2 \text{d}S =\iint y^2 \text{d}y\text{d}z  \\
& = & 
\int\limits_{-\pi}^{\pi}\int\limits_0^{R} \rho^2\cos^2\theta \rho \text{d}\rho \text{d}\theta \\
&=& \int\limits_{-\pi}^{\pi}\int\limits_0^{R} \dfrac{1+\cos 2\theta}{2} \rho \text{d}\rho^3 \text{d}\theta  \\
& = & \dfrac{1}{2} \int\limits_{-\pi}^{\pi} \left( 1+\cos 2\theta \right) \int\limits_0^{R} \rho^3 \text{d}\rho \text{d}\theta \\
& = & \dfrac{1}{8}R^4 \int\limits_{-\pi}^{\pi} \left( 1+\cos 2\theta \right)  \text{d}\theta \\
& = & \dfrac{1}{8}R^4 \left[\theta + \dfrac{1}{2} \sin 2\theta \right]_{-\pi}^{\pi} \\
& = & \dfrac{1}{8}R^4 \left(\pi + \dfrac{1}{2} \sin 2\pi  - \left(-\pi+ \dfrac{1}{2} \sin \left(-2\pi\right)  \right) \right) \\
& = & \dfrac{\pi R^4}{4} = \dfrac{\pi R^4}{4}=\dfrac{\pi D^4}{64}
\end{eqnarray*}
%
%\begin{eqnarray*}
%& = & \int \limits_{-R}^{R} \left[\int\limits_{-\sqrt{R^2-y^2}}^{\sqrt{R^2-y^2}} y^2 \text{d}z \right]\text{d}y \\%
%I_{Gz} &=& \iint y^2 \text{d}S =\int \limits_{-R}^{R} \left[\int\limits_{-\sqrt{R^2-y^2}}^{\sqrt{R^2-y^2}} y^2 \text{d}z \right]\text{d}y \\
%& = & \int \limits_{-R}^{R}  y^2  \left[ z\right] _{-\sqrt{R^2-y^2}}^{\sqrt{R^2-y^2}} \text{d}y  =\int \limits_{-R}^{R}  y^2  2\sqrt{R^2-y^2} \text{d}y  \\
%\end{eqnarray*}
%
%On a d'une part $y=\rho \cos\theta$  et $z=\rho\sin\theta$. D'autre part, 
%$\dfrac{\text{d}y}{\text{d}\theta}=-\rho \sin\theta$  et $\dfrac{\text{d}z}{\text{d}\theta}=\rho\cos\theta$.
%On a donc $\text{d}y \text{d}z =-\rho^2 \sin\theta\cos\theta $
\end{corrige}
\else 
\fi

\newpage


\section*{Exercice 2 -- Calcul de déformation}

\setcounter{subparagraph}{0}
On considère la poutre suivante soumise à une charge de densité linéique $p$ répartie uniformément :
\begin{center}
\includegraphics[width=.45\textwidth]{images/fig_02}
\end{center}
\subparagraph{}\textit{Déterminer les actions mécaniques dans les liaisons.}
\ifprof
\begin{corrige}
En utilisant le PFS, on montre que : $Y_A = Y_B = \dfrac{Lp}{2}$.
\end{corrige}
\else
\fi



\subparagraph{}\textit{Après avoir identifié les différentes parties et les différents tronçons, déterminer le torseur de cohésion.}
\ifprof
\begin{corrige}

Cette poutre n'est constituée que d'une seule partie. 

\begin{center}
\includegraphics[width=.45\textwidth]{images/fig_03}
\end{center}

On isole le tronçon $I$, on réalise le bilan des actions mécaniques et on applique le PFS : 
$$
\torseurscoh_{II \rightarrow I} + \left\{\mathcal{T}_{\text{Ext}\rightarrow I}\right\}  +\left\{\mathcal{T}_{\text{Pression}\rightarrow I} \right\}  
= \{0\}
$$

Détermination de $ \left\{\mathcal{T}_{\text{Ext}\rightarrow I}\right\}$ :

$\left\{\mathcal{T}_{\text{Ext}\rightarrow I}\right\} 
= \torseurl{\vectf{\text{Ext}}{I}=\dfrac{Lp}{2}\vect{y}}{\vectm{A}{\text{Ext}}{I}=\vect{0}}{A}
= \torseurl{\vectf{\text{Ext}}{I}=\dfrac{Lp}{2}\vect{y}}{\vectm{A}{\text{Ext}}{I}=-x\vect{x}\wedge \dfrac{Lp}{2}\vect{y} =- \dfrac{Lp}{2}x\vect{z} }{G} $


Détermination de $ \left\{\mathcal{T}_{\text{Pression}\rightarrow I}\right\}$ :

$\left\{\mathcal{T}_{\text{Pr}\rightarrow I}\right\} 
= \torseurl{\vectf{\text{Pr}}{I}=-px\vect{y}}{\vectm{G}{\text{Pr}}{I}=\dfrac{1}{2}px^2 \vect{z}}{G}
$

Détermination de $ \left\{\mathcal{T}_{\text{II}\rightarrow I}\right\}$ :

$
\left\{\mathcal{T}_{\text{II}\rightarrow I}\right\} 
= \torseurcol{N}{T_y}{T_z}{M_t}{M_{fy}}{M_{fz}}{G}
=  \torseurcol{0}{-\dfrac{Lp}{2}+px}{0}{0}{0}{\dfrac{Lp}{2}x-\dfrac{1}{2}px^2  }{G}
$


\end{corrige}
\else
\fi


\subparagraph{}\textit{Tracer les diagrammes des sollicitations.}
\ifprof
\begin{corrige}
~\\
\begin{center}
\includegraphics[width=.45\textwidth]{images/fig_04}
\end{center}

\end{corrige}
\else
\fi


\subparagraph{}\textit{Rechercher l'équation de déformation de la poutre ainsi que la flèche maximale.}
\ifprof
\begin{corrige}
L'équation de la déformée $y$ est donnée par : $EI_{Gz} y''(x)= M_{fz}(x)$. 

On a donc :

$
EI_{Gz} y''(x)= M_{fz}(x) 
\Leftrightarrow  EI_{Gz} y''(x)= \dfrac{Lp}{2}x-\dfrac{1}{2}px^2 
\Rightarrow  EI_{Gz} y'(x)= \dfrac{Lp}{2}\dfrac{1}{2}x^2 -\dfrac{1}{2}p\dfrac{1}{3}x^3$

$ \Rightarrow  EI_{Gz} y'(x)= \dfrac{Lp}{4}x^2 -\dfrac{1}{6}px^3 + v_0 $

$ \Rightarrow  EI_{Gz} y(x)= \dfrac{Lp}{12}x^3 -\dfrac{1}{24}px^4  + v_0 x +y_0$

Par ailleurs, la poutre étant en appui en $A$ et $B$, on a : $y(0) = 0$ et $y(L) = 0$.
En conséquences : 
$$
\left\{ 
\begin{array}{l}
0 = y_0 \\
0= \dfrac{Lp}{12}L^3 -\dfrac{1}{24}pL^4  + v_0 L +y_0
\end{array}
\right.
\Longleftrightarrow
\left\{ 
\begin{array}{l}
y_0 = 0 \\
0= \dfrac{p}{12}L^3 -\dfrac{1}{24}pL^3  + v_0 
\end{array}
\right.
\Longleftrightarrow
\left\{ 
\begin{array}{l}
y_0 = 0 \\
 v_0 = -\dfrac{p}{12}L^3 +\dfrac{1}{24}pL^3 =-\dfrac{pL^3}{24} 
\end{array}
\right.
$$


On a donc : 
$$
\left\{ 
\begin{array}{l}
  EI_{Gz} y'(x)= \dfrac{Lp}{4}x^2 -\dfrac{1}{6}px^3 -\dfrac{pL^3}{24} \\
  EI_{Gz} y(x)= \dfrac{Lp}{12}x^3 -\dfrac{1}{24}px^4  -\dfrac{pL^3}{24} x 
\end{array}
\right.
$$
La déformée maximale est atteinte lorsque la dérivée de la déformée est nulle :
$
  EI_{Gz} y'(x)= \dfrac{Lp}{4}x^2 -\dfrac{1}{6}px^3 -\dfrac{pL^3}{24}  = 0
$

Lorsque $x=L/2$, on a : -$y_{\text{max}}=-\dfrac{5}{384} \cdot \dfrac{pL^4}{EI}$

\end{corrige}
\else
\fi



\subparagraph{}\textit{Donner un modèle équivalent qui aurait pu être utilisé pour réaliser les calculs. }
\ifprof
\begin{corrige}
~\\
\begin{center}
\includegraphics[width=.45\textwidth]{images/fig_05}
\end{center}
\end{corrige}
\else
\fi




%\end{multicols}

\end{document}


