\documentclass[10pt,fleqn]{article} % Default font size and left-justified equations
\usepackage[%
    pdftitle={RdM : Diagrammes de sollicitations},
    pdfauthor={Xavier Pessoles}]{hyperref}
    
\input{style/new_style}
\input{style/macros_SII}

\usepackage{multicol}
\usepackage{style/schemabloc}
\fichetrue
%\fichefalse

\proftrue
%\proffalse

\tdtrue
%\tdfalse

%\courstrue
\coursfalse

% -------------------------------------
% Déclaration des titres
% -------------------------------------

\def\discipline{Sciences \\Industrielles de \\ l'Ingénieur}
\def\xxtete{Sciences Industrielles de l'Ingénieur}

\def\classe{Colle}
\def\xxnumpartie{Cycle RdM}
\def\xxpartie{Résistance des matériaux}

\def\xxnumchapitre{Chapitre 4 \vspace{.2cm}}
\def\xxchapitre{\hspace{.12cm} Modélisation des pièces déformables en flexion}

\def\xxposongletx{2}
\def\xxposonglettext{1.45}
\def\xxposonglety{19}%16

\def\xxonglet{Cycle RdM -- Ch. 4}

\def\xxactivite{TD -- Colle}
\def\xxtitreexo{Création de motif sur de la poudre de maquillage compactée}
\def\xxsourceexo{Concours Centrale Supelec TSI 2016}
\def\xxauteur{\textsl{Xavier Pessoles}}

\def\xxcompetences{%
\textsl{%
\textbf{Savoirs et compétences :}\\
}}

\def\xxfigures{
%incgraphics[width=.8\textwidth]{}%images/prot_01}
}%figues de la page de garde

\def\xxpied{%
Cycle RdM -- Modélisation des pièces déformables\\
Flexion -- \xxactivite%
}

\setcounter{secnumdepth}{5}
%---------------------------------------------------------------------------

\usepackage{siunitx}
\begin{document}
%\chapterimage{png/Fond_Cin}
\input{style/new_pagegarde}
\vspace{10cm}
\pagestyle{fancy}
\thispagestyle{plain}


\def\columnseprulecolor{\color{ocre}}
\setlength{\columnseprule}{0.4pt} 

\begin{multicols}{2}

\subsection*{Influence des déformations des constituants de l’axe sur la précision
de positionnement des godets par rapport aux buses}

\begin{obj}
L’objectif de cette partie est de vérifier que la précision du positionnement du godet par rapport à la
buse est respectée en dépit de la déformation des profilés de guidage de l’axe $\vect{x}$. On souhaite que la
buse soit positionnée à 0,08 mm près (exigence 1.1 du diagramme des exigences).
\end{obj}

%Les profilés sont modélisés comme des poutres parallèles à la direction 𝑥.⃗ La position du centre de gravité de
%la partie mobile 2 est notée 𝐺2. On se place dans le cas où le chariot accélère le long de l’axe 𝑥⃗ : 𝑎u⃗ 2,2/0 = 𝛾𝑥⃗
%avec 𝛾 = –10 m⋅s–2 est l’accélération du chariot par rapport au bâti 0 lié à la Terre. Un référentiel lié au bâti
%est supposé galiléen. Cette accélération est assurée par l’action mécanique des courroies sur le chariot 2 et
%2016-03-02 14:08:28 Page 11/12
%qui n’encaissent aucun moment. L’action mécanique des courroies sur le chariot mobile 2 est modélisée par le
%torseur : 𝑇courroies→2 =
%u(ℎ)
%⎧{⎨{⎩
%𝑋u 0
%0 0
%0 0
%⎫}⎬}⎭
%(u⃗,u,⃗ u)⃗
%.
%Notations et paramétrage On considère que le chariot est disposé sur le plan de symétrie des deux profilés
%constituant l’axe 𝑥.⃗ De ce fait, le système est considéré comme plan dans le plan (𝑂, 𝑥,⃗ 𝑦)⃗ . De même, les deux
%profilés sont assimilés à un seul dont la section droite est notée 𝑆 et le moment quadratique autour de l’axe
%(𝐺, 𝑧)⃗ est noté 𝐼. On note la longueur 𝑂𝐴 = 𝐿 = 304 mm. Le chariot est positionné en 𝐻(ℎ). La position du
%centre de gravité 𝐺2 du chariot 2 est telle que ⃗𝐻⃗⃗⃗⃗⃗⃗⃗𝐺⃗⃗⃗⃗⃗⃗2 = 𝑎𝑥⃗ + 𝑏𝑦⃗ avec 𝑎 = 180 mm et 𝑏 = 100 mm. La position
%de la buse est en 𝐵 tel que ⃗𝐻⃗⃗⃗⃗⃗⃗⃗𝐵⃗⃗⃗⃗⃗⃗ = 𝑎𝑥⃗ + 𝑐𝑦⃗ avec 𝑐 = 230 mm. La masse du chariot 2 est notée 𝑀2 = 25 kg.
%L’accélération de la pesanteur est notée 𝑔 = 9,8 m⋅s–2.
\end{multicols}

\end{document}


