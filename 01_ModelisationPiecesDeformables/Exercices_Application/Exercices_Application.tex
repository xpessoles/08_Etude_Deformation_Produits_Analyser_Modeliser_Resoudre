\documentclass[10pt,fleqn]{article} % Default font size and left-justified equations
\usepackage[%
    pdftitle={RdM : Diagrammes de sollicitations},
    pdfauthor={Xavier Pessoles}]{hyperref}
    
\input{style/new_style}
\input{style/macros_SII}

\usepackage{multicol}
\usepackage{style/schemabloc}
\fichetrue
%\fichefalse

\proftrue
\proffalse

\tdtrue
%\tdfalse

%\courstrue
\coursfalse

\def\discipline{Sciences \\Industrielles de \\ l'Ingénieur}
\def\xxtete{Sciences Industrielles de l'Ingénieur}

\def\classe{PT -- PT$\star$}
\def\xxnumpartie{Cycle 2}
\def\xxpartie{Modélisation des sollicitations dans un solide déformable et mesure des déformations.}

\def\xxnumchapitre{}%Chapitre n}
\def\xxchapitre{}%Titre Chapitre}

\def\xxtitreexo{Exercices d'application -- Détermination du torseur de cohésion.}
\def\xxsourceexo{}%\hspace{.2cm} D'après notes de cours PT -- Lycée G. Eiffel, Bordeaux.}


\def\xxposongletx{2}
\def\xxposonglettext{1.45}
\def\xxposonglety{20}
\def\xxonglet{Cycle 2}

\def\xxactivite{Applications}
\def\xxauteur{\textsl{Équipe pédagogique PT -- PT$\star$}}

\def\xxcompetences{%
\textsl{%
\textbf{Savoirs et compétences :}\\
\noindent% \textbf{Résoudre :} à partir des modèles retenus :
\begin{itemize}[label=\ding{112},font=\color{ocre}] 
\item Mod2-C16-S1	: Déterminer le torseur de cohésion dans un solide
\item Mod2-C16-S2	 : Identifier les sollicitations (traction, compression, flexion, torsion, cisaillement)
\end{itemize}
%
%\noindent \textit{Mod2 -- C4.1 :} Représentation par schéma bloc.
}}

\def\xxfigures{
\includegraphics[width=.8\textwidth]{images/rdm}
}%figues de la page de garde

\def\xxpied{%
Cycle 2 -- Modélisation des sollicitations \\
%Ch. 2 : Modélisation des Systèmes Linéaires Continus Invariants -- Transformée de Laplace -- 
\xxactivite%
}


\setcounter{secnumdepth}{5}
%---------------------------------------------------------------------------


\begin{document}
%\chapterimage{png/Fond_Cin}
\input{style/new_pagegarde}
\vspace{7cm}
\pagestyle{fancy}
\thispagestyle{plain}


\def\columnseprulecolor{\color{ocre}}
\setlength{\columnseprule}{0.4pt} 
\ifprof
\else
\begin{multicols}{2}
\fi

\section*{Exercice 1}
\setcounter{subparagraph}{0}
On donne sur le schéma ci-dessous la modélisation d'une poutre et des efforts qui lui sont appliqués.
\begin{center}
\includegraphics[width=.45\textwidth]{images/exo_08}
\end{center}

\subparagraph{}
\textit{Proposer une méthode permettant de déterminer le torseur de cohésion dans chacun des tronçons. Est-il nécessaire de déterminer les actions mécaniques en $A$ ?}

\subparagraph{}
\textit{Déterminer le torseur de cohésion dans chacun des tronçons.}


\ifprof
\begin{corrige}
\subsubsection*{On considère le tronçon AB}
On considère donc $\vect{AG} = \lambda \vect{x}$ avec $\lambda\in \left[0,\dfrac{L}{2}\right]$.
\begin{center}
\includegraphics[width=.45\textwidth]{images/exo_08_corr_01}
\end{center}


On isole la partie de droite, notée $+$ soumise aux actions de :
\begin{itemize}[label=\ding{112},font=\color{ocre}] 
\item torseur de cohésion : $\torseurscoh_{S-\rightarrow S+}$ en $G$;
\item action mécanique en $B$ : $\torseurcol{0}{-F_1}{0}{0}{0}{0}{B,\left( \vect{x},\vect{y},\vect{z} \right)}  =\torseurcol{0}{-F_1}{0}{0}{0}{-\left(\dfrac{L}{2}-\lambda \right)F_1}{G,\left( \vect{x},\vect{y},\vect{z} \right)} $;
\item action mécanique en $D$ : $\torseurcol{F_2}{0}{0}{0}{0}{0}{D,\left( \vect{x},\vect{y},\vect{z} \right)}  =\torseurcol{F_2}{0}{0}{0}{0}{-F_2h}{G,\left( \vect{x},\vect{y},\vect{z} \right)} $.
\end{itemize}


Par application du PFS on a : 
$$ \torseurscoh_{S-\rightarrow S+} +\torseurstat{T}{F_1}{S+} +\torseurstat{T}{F_2}{S+} = \{0\} \Leftrightarrow \torseurscoh_{S+\rightarrow S-} =\torseurstat{T}{F_1}{S+} +\torseurstat{T}{F_2}{S+}  $$
On a donc  :
$$
\torseurcol{N}{T_y}{T_z}{M_t}{Mf_y}{Mf_z}{G} = 
\torseurcol{F_2}{-F_1}{0}{0}{0}{-\left(\dfrac{L}{2}-\lambda \right)F_1-F_2h}{G,\left( \vect{x},\vect{y},\vect{z} \right)}
$$

\subsubsection*{On considère le tronçon BC}
On considère donc $\vect{AG} = \lambda \vect{x}$ avec $\lambda\in \left[\dfrac{L}{2},L\right]$.
\begin{center}
\includegraphics[width=.45\textwidth]{images/exo_08_corr_02}
\end{center}


On isole la partie de droite, notée $+$ soumise aux actions de :
\begin{itemize}[label=\ding{112},font=\color{ocre}] 
\item torseur de cohésion : $\torseurscoh_{S-\rightarrow S+}$ en $G$;
\item action mécanique en $D$ : $\torseurcol{F_2}{0}{0}{0}{0}{0}{D,\left( \vect{x},\vect{y},\vect{z} \right)}  =\torseurcol{F_2}{0}{0}{0}{0}{-F_2h}{G,\left( \vect{x},\vect{y},\vect{z} \right)} $.
\end{itemize}


Par application du PFS on a : 
$$ \torseurscoh_{S-\rightarrow S+} +\torseurstat{T}{F_2}{S+} = \{0\} \Leftrightarrow \torseurscoh_{S+\rightarrow S-} =\torseurstat{T}{F_2}{S+}  $$
On a donc  :
$$
\torseurcol{N}{T_y}{T_z}{M_t}{Mf_y}{Mf_z}{G} = 
\torseurcol{F_2}{0}{0}{0}{0}{-F_2h}{G,\left( \vect{x},\vect{y},\vect{z} \right)}
$$
\subsubsection*{On considère le tronçon CD}
On considère donc $\vect{AG} = \lambda \vect{x}_1-L \vect{y}_1$  avec $\lambda\in \left[0,h\right]$.
\begin{center}
\includegraphics[width=.45\textwidth]{images/exo_08_corr_03}
\end{center}


On isole la partie de droite, notée $+$ soumise aux actions de :
\begin{itemize}[label=\ding{112},font=\color{ocre}] 
\item torseur de cohésion : $\torseurscoh_{S-\rightarrow S+}$ en $G$;
\item action mécanique en $D$ : $\torseurcol{0}{-F_2}{0}{0}{0}{0}{D,\left( \vect{x_1},\vect{y_1},\vect{z} \right)}  =\torseurcol{0}{-F_2}{0}{0}{0}{-F_2\left(h-\lambda\right)}{G,\left( \vect{x_1},\vect{y_1},\vect{z} \right)} $.
\end{itemize}


Par application du PFS on a : 
$$ \torseurscoh_{S-\rightarrow S+} +\torseurstat{T}{F_2}{S+} = \{0\} \Leftrightarrow \torseurscoh_{S+\rightarrow S-} =+\torseurstat{T}{F_2}{S+}  $$
On a donc  :
$$
\torseurcol{N}{T_y}{T_z}{M_t}{Mf_y}{Mf_z}{G,\left( \vect{x_1},\vect{y_1},\vect{z} \right)} = 
\torseurcol{0}{-F_2}{0}{0}{0}{-F_2\left(h-\lambda\right)}{G,\left( \vect{x_1},\vect{y_1},\vect{z} \right)}
$$



\end{corrige}
\else 
\fi


\subparagraph{}
\textit{Tracer le diagramme des sollicitations.}



\section*{Exercice 2}
\setcounter{subparagraph}{0}
On donne sur le schéma ci-dessous la modélisation d'une poutre et des efforts qui lui sont appliqués.
\subparagraph{}
\textit{Est-il nécessaire de déterminer les actions mécaniques en $A$ et en $B$.}
\ifprof
\begin{corrige}
Si on isole la partie <<II>>, elle est soumise à l'effort $\vect{F}$ et à l'action du torseur de cohésion. On n'aura donc pas besoin de l'action dans la liaison encastrement pour pouvoir déterminer le torseur de cohésion. 
\end{corrige}
\else 
\fi


On cherche à déterminer le diagramme des sollicitations dans chacun des tronçons.

\subparagraph{}
\textit{Exprimer le torseur de cohésion dans chacun des tronçons.}
\ifprof
\begin{corrige}
 ~\\
\begin{itemize}[label=\ding{112},font=\color{ocre}] 
\item On isole la portion $[MB]$ (+).
\item La portion est soumise d'une part à l'action mécanique en $B$ et d'autre part à l'action mécanique du torseur de cohésion.
\item On a donc : 
\end{itemize}
$$
\torseurscoh_{S-\rightarrow S+} +
 \torseurcol{-F}{0}{0}{0}{0}{0}{B,\left( \vect{x},\vect{y},\vect{z} \right)} 
 = \{0\}
$$
%
%$$
%\torseurscoh_{S+\rightarrow S-}
%=-\torseurcol{F}{0}{0}{0}{0}{0}{G}
%-\torseurcol{0}{-px}{0}{0}{0}{ \dfrac{px^2}{2}}{G}
%=\torseurcol{-F}{px}{0}{0}{0}{-\dfrac{px^2}{2}}{G}
%$$
%car 
$\vectm{M}{\text{Ext}}{\text{Poutre}}
=\vectm{B}{\text{Ext}}{\text{Poutre}} + \vect{MB}  \wedge -F \vect{x} 
=\left( -R\vect{u} + R\vect{x}\right) \wedge -F \vect{x} 
=  - RF \sin \theta \vect{z}
$  et $-F \vect{x}  = -F \left( \cos \theta \vect{u}-\sin \theta \vect{v} \right) $.

On a donc :
$$
\torseurscoh_{S+\rightarrow S-} 
= 
 \torseurcol{-F\cos \theta }{F\sin \theta }{0}{0}{0}{ - RF \sin \theta}{M,\left( \vect{u},\vect{v},\vect{z} \right)} 
$$

\textbf{TODO : diagrammes}
\end{corrige}
\else 
\fi


%\subparagraph{}
%\textit{Tracer les diagrammes des sollicitations.}
%\ifprof
%\begin{corrige}
%\end{corrige}
%\else 
%\fi
\ifprof
\else
\begin{center}
\includegraphics[width=.4\textwidth]{images/exo_04}
\end{center}
\fi


\section*{Exercice 3}
\setcounter{subparagraph}{0}
On donne sur le schéma ci-dessous la modélisation d'une poutre et des efforts qui lui sont appliqués. On note $p$ la densité d'effort linéique.
\begin{center}
\includegraphics[width=.45\textwidth]{images/exo_05}
\end{center}

\subparagraph{}
\textit{Déterminer les actions mécaniques en $A$ et en $B$.}
\ifprof
\begin{corrige}
On a $Y_A = Y_B = \dfrac{pl}{2}$.
\end{corrige}
\else 
\fi


%On cherche à déterminer le diagramme des sollicitations dans chacun des tronçons.

\subparagraph{}
\textit{Exprimer le torseur de cohésion dans chacun des tronçons.}
\ifprof
\begin{corrige}
\begin{itemize}
\item On isole la portion $[GB]$.
\item La portion est soumise à l'action mécanique en $B$, à l'action uniformément répartie (exprimée en $M$, milieu de $[GB]$) et à l'action mécanique du torseur de cohésion en $G$.
\item On a donc : 
\end{itemize}
$$
\torseurscoh_{S-\rightarrow S+} +
 \torseurcol{0}{\dfrac{pl}{2}}{0}{0}{0}{0}{B,\left( \vect{x},\vect{y},\vect{z} \right)} 
 +\torseurcol{0}{-p(l-x)}{0}{0}{0}{0}{M,\left( \vect{x},\vect{y},\vect{z} \right)} 
 = \{0\}
$$
On a donc, $\forall x \in \left[0,l\right]$, 
$$
\torseurscoh_{S+\rightarrow S-}
= \torseurcol{0}{\dfrac{pl}{2}}{0}{0}{0}{(l-x)\dfrac{pl}{2}}{G} 
 +\torseurcol{0}{-p(l-x)}{0}{0}{0}{-p\dfrac{\left(l-x\right)^2}{2}}{G} 
 = \torseurcol{0}{\dfrac{pl}{2}-p(l-x)}{0}{0}{0}{(l-x)\dfrac{pl}{2}-p\dfrac{\left(l-x\right)^2}{2}}{G} 
$$
car 
$\vectm{G}{\text{Ext}}{\text{Poutre}}
=\vectm{B}{\text{Ext}}{\text{Poutre}} + \vect{GB}\wedge \dfrac{pl}{2} \vect{y}  
=  (l-x)\vect{x}  \wedge \dfrac{pl}{2} \vect{y}
=  \dfrac{pl(l-x)}{2} \vect{z}
$ 

et 
$\vectm{G}{\text{Ext}}{\text{Poutre}}
=  \vectm{M}{\text{Ext}}{\text{Poutre}} + \vect{GM}\wedge \left( -p(l-x) \right) \vect{y}  
= \left(\dfrac{l-x}{2}\right)\vect{x}\wedge \left(-p(l-x)\right) \vect{y}  
=  -p\dfrac{\left(l-x\right)^2}{2}\vect{z}  
$.
\end{corrige}
\else 
\fi

\subparagraph{}
\textit{Tracer les diagrammes des sollicitations.}
\ifprof
\begin{corrige}
\end{corrige}
\else 
\fi

%\newpage

\section*{Exercice 5}
\setcounter{subparagraph}{0}
On donne sur le schéma ci-dessous la modélisation d'une poutre. On y exerce une charge répartie de pression $p$ (en $\text{N}\text{m}^{-1}$).
\begin{center}
\includegraphics[width=.45\textwidth]{images/exo_06}
\end{center}

\subparagraph{}
\textit{Déterminer les actions mécaniques en $A$ et en $B$.}
\ifprof
\begin{corrige}
L'action de pression est modélisable par le glisseur suivant : 
$$\mathcal{T}_{\text{Pression}\rightarrow \text{Poutre}} =\torseurl{\vectf{\text{Pr}}{\text{Po}}=- 2 p R \vect{y}=-F\vect{y}}{\vectm{O}{\text{Pr}}{\text{Po}}=\vect{0}}{O}$$

On a donc :
$$\mathcal{T}_{\text{Ext}\rightarrow \text{Poutre A}} =\torseurl{\vectf{\text{Ext}}{\text{Po A}}=\dfrac{1}{2}F\vect{y}}{\vectm{O}{\text{Pr}}{\text{Po A}}=\vect{0}}{O}$$

$$\mathcal{T}_{\text{Ext}\rightarrow \text{Poutre B}} =\torseurl{\vectf{\text{Ext}}{\text{Po B}}=\dfrac{1}{2}F\vect{y}}{\vectm{O}{\text{Pr}}{\text{Po B}}=\vect{0}}{O}$$



\end{corrige}
\else 
\fi


On cherche à déterminer le diagramme des sollicitations dans chacun des tronçons.

\subparagraph{}
\textit{Quels tronçons peut-on considérer ?}
\ifprof
\begin{corrige}
On ne considèrera qu'une seule partie, dans laquelle on aura deux tronçons.

\begin{center}
\includegraphics[width=.9\textwidth]{images/exo_06_param}
\end{center}

\end{corrige}
\else 
\fi

\subparagraph{}
\textit{Exprimer le torseur de cohésion dans chacun des tronçons, de préférence dans une base locale.}

\ifprof
\begin{corrige}
Pour $\theta \in \left[0 ;  \pi \right]$, en appliquant le théorème de la résultante statique sur le tronçon $I$ on a : 
%$$
%\torseurscoh_{II \rightarrow I} 
%+\left\{\mathcal{T}_{\text{Ext}\rightarrow \text{Poutre B}}\right\} 
%+\left\{\mathcal{T}_{\text{Pression}\rightarrow \text{Poutre}} \right\}  = \{0\}
%$$

$$
\torseurscoh_{II \rightarrow I} 
+\left\{\mathcal{T}_{\text{Ext}\rightarrow I}\right\} 
+\left\{\mathcal{T}_{\text{Pression}\rightarrow I} \right\}  = \{0\}
$$
\end{corrige}

\begin{corrige}
Exprimons $\left\{\mathcal{T}_{\text{Ext}\rightarrow \text{I}}\right\} $ au point $P$ : 

$$\left\{\mathcal{T}_{\text{Ext}\rightarrow \text{I}}\right\} 
=\torseurl{\vectf{\text{Ext}}{\text{I}} }{\vectm{P}{\text{Ext}}{\text{I}} }{P}
$$

\begin{itemize}
\item $\vectf{\text{Ext}}{\text{I}} =\dfrac{1}{2}F\vect{y} = \dfrac{1}{2}F\left( \sin\theta \vect{e_r} + \cos\theta \vect{e_\theta} \right) $ 

$\vectf{\text{Ext}}{\text{I}} = pR\left( \sin\theta \vect{e_r} + \cos\theta \vect{e_\theta} \right) $ 

\item $\vectm{P}{\text{Ext}}{\text{I}} $


$= \vectm{B}{\text{Ext}}{\text{I}} + \vect{PB} \wedge \vectf{\text{Ext}}{\text{I}}$

$= \left( -R \vect{e_r} +R \vect{x} \right) \wedge\dfrac{1}{2}F\vect{y}$

$= \left( -R \cos \theta \vect{x} -R \sin \theta \vect{y} +R \vect{x} \right) \wedge\dfrac{1}{2}F\vect{y}$

$= \dfrac{1}{2}F \left( -R \cos \theta  +R \right)\vect{z}$

$= \dfrac{RF}{2} \left( 1- \cos \theta \right)\vect{z}$

$= pR^2 \left( 1- \cos \theta \right)\vect{z}$.


$$\left\{\mathcal{T}_{\text{Ext}\rightarrow \text{I}}\right\} 
=\torseurl{ 
 pR\left( \sin\theta \vect{e_r} + \cos\theta \vect{e_\theta} \right)}{
pR^2 \left( 1- \cos \theta \right)\vect{z}} {P}
$$

%\item $\vectm{O}{\text{Ext}}{\text{I}} $
%
%
%$= \vectm{B}{\text{Ext}}{\text{I}} + \vect{OB} \wedge \vectf{\text{Ext}}{\text{I}}$
%
%$=  R \vect{x}  \wedge\dfrac{1}{2}F\vect{y}=\dfrac{RF}{2}\vect{z}$.
\end{itemize}

\end{corrige}

\begin{corrige}
Exprimons $\left\{\mathcal{T}_{\text{Pression}\rightarrow \text{I}}\right\} $ au point $O$.

$$\left\{\mathcal{T}_{\text{Pr}\rightarrow \text{I}}\right\} 
=\torseurl{\vectf{\text{Pr}}{\text{I}} }{\vectm{O}{\text{Pr}}{\text{I}} = \vect{0} }{O} 
$$

\begin{itemize}
\item $\vectf{\text{Pr}}{\text{I}} = \int_0^{\theta} pR (-\vect{e_r})\text{d}\alpha $

$= -pR \left( \left[ \sin \alpha \right]_0^\theta \vect{x} - \left[ \cos\alpha  \right]_0^\theta \vect{y}\right)$

$= -pR \left( \sin \theta \vect{x} - \left( \cos\theta -1  \right) \vect{y}\right)$

$= -pR \left( \sin \theta \left( \cos\theta\vect{e_r} - \sin\theta \vect{e_\theta} \right) \right. $

$\left.- \left( \cos\theta -1  \right) \left(  \cos\theta\vect{e_\theta} + \sin\theta \vect{e_r} \right)\right)$

$= -pR \left( \sin \theta\cos\theta\vect{e_r} - \sin^2\theta \vect{e_\theta}  +    \cos\theta\vect{e_\theta} \right.$

$\left. + \sin\theta \vect{e_r} - \cos^2\theta\vect{e_\theta} - \cos\theta\sin\theta \vect{e_r} \right)$

$= -pR \left(   \sin\theta \vect{e_r}  +\left( \cos\theta -1 \right) \vect{e_\theta}  \right)$



$$\left\{\mathcal{T}_{\text{Pr}\rightarrow \text{I}}\right\} 
=\torseurl{-pR \left(   \sin\theta \vect{e_r}  +\left( \cos\theta -1 \right) \vect{e_\theta}  \right) }{\vect{0} }{O} 
$$

%$= pR \left( - \sin ^2\theta \vect{e_r}  -\cos^2 \theta\vect{e_\theta}  + \cos\theta \vect{e_\theta} + \sin\theta \vect{e_r} \right)$


%
%$- pR\theta \cos\dfrac{\theta}{2} \vect{x}- pR\theta \sin\dfrac{\theta}{2} \vect{y}$
%
%$=- pR\theta \cos\dfrac{\theta}{2} \left(\cos \theta  \vect{e_r} - \sin \theta  \vect{e_\theta}   \right) $
%
%$- pR\theta \sin\dfrac{\theta}{2}  \left( \cos \theta  \vect{e_\theta} +\sin \theta  \vect{e_r}   \right) $
%
%$ = - pR\theta \cos\dfrac{\theta}{2} \cos \theta  \vect{e_r} + pR\theta \cos\dfrac{\theta}{2} \sin \theta  \vect{e_\theta}   $
%
%$
%- pR\theta \sin\dfrac{\theta}{2} \cos \theta  \vect{e_\theta} 
%- pR\theta \sin\dfrac{\theta}{2} \sin \theta  \vect{e_r}  
%$
%
%$
%= - pR\theta\left(\cos\dfrac{\theta}{2} \cos \theta   + \sin\dfrac{\theta}{2} \sin \theta \right) \vect{e_r}
%$
%
%$
%+pR\theta  \left( \cos\dfrac{\theta}{2} \sin \theta  
%- \sin\dfrac{\theta}{2} \cos \theta    \right)\vect{e_\theta}
%$
%
%$
%= - pR\theta\left(\cos\dfrac{\theta}{2} \right) \vect{e_r}
%+pR\theta  \left( \sin\dfrac{\theta}{2}  \right)\vect{e_\theta}
%$

\end{itemize}
\end{corrige}

\begin{corrige}
Exprimons $\left\{\mathcal{T}_{\text{Pression}\rightarrow \text{I}}\right\} $ au point $P$.

$$\left\{\mathcal{T}_{\text{Pr}\rightarrow \text{I}}\right\} 
=\torseurl{\vectf{\text{Pr}}{\text{I}} }{\vectm{P}{\text{Pr}}{\text{I}} }{P} 
$$


$\vectm{P}{\text{Pr}}{\text{I}} = \vectm{O}{\text{Pr}}{I} + \vect{PO} \wedge\vectf{\text{Pr}}{\text{I}} $

$ = -R\vect{e_r}\wedge \left(  -pR \left(   \sin\theta \vect{e_r}  +\left( \cos\theta -1 \right) \vect{e_\theta}  \right)\right)$

$ = pR^2 \left( \cos\theta -1 \right) \vect{z}$




$$\left\{\mathcal{T}_{\text{Pr}\rightarrow \text{I}}\right\} 
=\torseurl{-pR \left(   \sin\theta \vect{e_r}  +\left( \cos\theta -1 \right) \vect{e_\theta}  \right) }{pR^2 \left( \cos\theta -1 \right) \vect{z} }{P} 
$$

\end{corrige}





\begin{corrige}

En conclusion, 
$$
\torseurscoh_{II \rightarrow I} 
+\left\{\mathcal{T}_{\text{Ext}\rightarrow I}\right\} 
+\left\{\mathcal{T}_{\text{Pression}\rightarrow I} \right\}  = \{0\}
$$

$$
\torseurscoh_{II \rightarrow I}
= 
-\left\{\mathcal{T}_{\text{Ext}\rightarrow I}\right\} 
-\left\{\mathcal{T}_{\text{Pression}\rightarrow I} \right\}  
$$

$$
\torseurscoh_{II \rightarrow I}
= 
-\torseurl{ 
 pR\left( \sin\theta \vect{e_r} + \cos\theta \vect{e_\theta} \right)}{
pR^2 \left( 1- \cos \theta \right)\vect{z}} {P}
-\torseurl{-pR \left(   \sin\theta \vect{e_r}  +\left( \cos\theta -1 \right) \vect{e_\theta}  \right) }{pR^2 \left( \cos\theta -1 \right) \vect{z} }{P} 
$$



$$
\torseurscoh_{II \rightarrow I}
= 
\torseurl{ 
 -pR\left( \sin\theta \vect{e_r} + \cos\theta \vect{e_\theta} \right) +pR \left(   \sin\theta \vect{e_r}  +\left( \cos\theta -1 \right) \vect{e_\theta}  \right) }{
-pR^2 \left( 1- \cos \theta \right)\vect{z} - pR^2 \left( \cos\theta -1 \right) } {P}
$$


$$
\torseurscoh_{II \rightarrow I}
= 
\torseurl{ 
-pR  \vect{e_\theta}  }{
\vect{0} } {P}
$$

\end{corrige}
\else
\fi





\newpage



\section*{Exercice 1}
\setcounter{subparagraph}{0}
On donne sur le schéma ci-dessous la modélisation d'une poutre et des efforts qui lui sont appliqués.
\begin{center}
\includegraphics[width=.45\textwidth]{images/exo_01}
\end{center}

\subparagraph{}
\textit{Déterminer les actions mécaniques en $A$ et en $B$.}
\ifprof
\begin{corrige}~\\

\begin{itemize}[label=\ding{112},font=\color{ocre}] 
\item On isole la poutre.
\item On réalise le bilan des actions mécaniques :
\begin{itemize}[label=\ding{110},font=\color{ocre} \footnotesize] 
\item liaison sphère-plan en $O$. Sans frottement, cette action est de direction $\vect{y}$;
\item liaison sphère-plan en $C$. Sans frottement, cette action est de direction $\vect{y}$;
\item action mécanique en $B$.
\end{itemize}
\item On réalise un théorème de la résultante statique en $C$ en projection sur $\vect{y}$ et un théorème du moment statique appliqué au point $O$ en projection suivant $\vect{z}$ :
$$
\left\{
\begin{array}{l}
Y_O + Y_C -F = 0 \\
-\dfrac{3l}{4}F + l Y_C = 0 \\
\end{array}
\right.
\Leftrightarrow
\left\{
\begin{array}{l}
Y_O = F -  \dfrac{3}{4}F  = \dfrac{F}{4} \\
Y_C = \dfrac{3}{4}F \\
\end{array}
\right.
$$
\end{itemize}

\end{corrige}
\else 
\fi
On cherche à déterminer le diagramme des sollicitations dans chacun des tronçons.

\subparagraph{}
\textit{Quels tronçons peut-on considérer ?}
\ifprof
\begin{corrige}
Dans le cadre de cette étude on considèrera les tronçons suivants : $x\in\left[0,\dfrac{3l}{4}\right]$ et $x\in\left[\dfrac{3l}{4},l\right]$.

\end{corrige}
\else 
\fi

\subparagraph{}
\textit{Exprimer le torseur de cohésion dans chacun des tronçons.}
\ifprof
\begin{corrige}~\\
\begin{itemize}[label=\ding{112},font=\color{ocre}] 
\item On isole le premier tronçon.
\item Le tronçon est soumis d'une part à l'action mécanique en $O$ et d'autre part à l'action mécanique du torseur de cohésion.
\item On a donc :
\end{itemize}
$$
\torseurscoh_{S+\rightarrow S-} + \torseurcol{0}{Y_O}{0}{0}{0}{0}{A,\left( \vect{x},\vect{y},\vect{z} \right)} = \{0\}
$$
On a donc :
$$\torseurscoh_{S+\rightarrow S-}
=-\torseurcol{0}{Y_O}{0}{0}{0}{0}{A}
=\torseurcol{0}{-\dfrac{F}{4}}{0}{0}{0}{ \dfrac{F}{4}x}{G}
$$
car $\vectm{G}{\text{Ext}}{\text{Poutre}}=\vectm{O}{\text{Ext}}{\text{Poutre}} + \vect{GO}\wedge Y_O \vect{y}  = -x\vect{x} \wedge Y_O \vect{y} = -x Y_O \vect{z}= -x \dfrac{F}{4} \vect{z}$.
%\end{corrige}

%\begin{corrige}
\begin{itemize}[label=\ding{112},font=\color{ocre}] 
\item On isole le second tronçon.
\item Le tronçon est soumis d'une part à l'action mécanique en $C$ et d'autre part à l'action mécanique du torseur de cohésion.
\item On a donc :
\end{itemize}
$$
\torseurscoh_{S-\rightarrow S+} + \torseurcol{0}{Y_C}{0}{0}{0}{0}{A,\left( \vect{x},\vect{y},\vect{z} \right)} = \{0\}
$$
On a donc, $\forall x \in\left[\dfrac{3l}{4},l\right]$ :
$$
\torseurscoh_{S-\rightarrow S+}
=-\torseurcol{0}{Y_C}{0}{0}{0}{0}{A}
=-\torseurcol{0}{\dfrac{3F}{4}}{0}{0}{0}{ (l-x) \dfrac{3F}{4}}{G}
$$
car $\vectm{G}{\text{Ext}}{\text{Poutre}}=\vectm{C}{\text{Ext}}{\text{Poutre}} + \vect{GC}\wedge Y_C \vect{y}  = (l-x)\vect{x} \wedge Y_C \vect{y} = (l-x) Y_C \vect{z}= (l-x) \dfrac{3F}{4} \vect{z}$.

Au final,
$$\torseurscoh_{S+\rightarrow S-}
=\torseurcol{0}{\dfrac{3F}{4}}{0}{0}{0}{ (l-x) \dfrac{3F}{4}}{G}
$$
\end{corrige}
\else 
\fi

\subparagraph{}
\textit{Tracer les diagrammes des sollicitations.}
\ifprof
\begin{corrige}~\\

\begin{center}
\includegraphics[width=.5\textwidth]{images/exo_01_corr}
\end{center}
\end{corrige}
\else 
\fi



\section*{Exercice 2}
\setcounter{subparagraph}{0}
On donne sur le schéma ci-dessous la modélisation d'une poutre et des efforts qui lui sont appliqués.
\begin{center}
\includegraphics[width=.45\textwidth]{images/exo_01_01}
\end{center}

\subparagraph{}
\textit{Donner une méthode permettant de déterminer le torseur de cohésion sans calculer les actions mécaniques en $O$.}
\ifprof
\begin{corrige} Si on isole la partie comprise entre $G$ et $A$ et qu'on applique le PFS, cette partie est soumise aux efforts de cohésion et à l'action mécanique en $A$. Il n'est donc pas nécessaire de déterminer les efforts en $O$.
\end{corrige}
\else
\fi

\subparagraph{}
\textit{Exprimer le torseur de cohésion.}
\ifprof
\begin{corrige} ~\\
\begin{minipage}[c]{.48\linewidth}
\begin{itemize}[label=\ding{112},font=\color{ocre}] 
\item On isole la portion comprise entre $G$ et $A$.
\item Cette partie est soumises aux actions mécaniques de l'effort en $A$ et des actions du torseur de cohésion. 
\item On réalise le PFS sur cette partie et on a :
\end{itemize}
\end{minipage} \hfill
\begin{minipage}[c]{.48\linewidth}
\begin{center}
\includegraphics[width=.85\linewidth]{images/exo_01_01_corr_01}
\end{center}
\end{minipage} 

$$
\torseurscoh_{S-\rightarrow S+} + \torseurcol{0}{0}{-F}{0}{0}{0}{A,\left( \vect{x},\vect{y},\vect{z} \right)} = \{0\}
$$
On a donc, $\forall x \in\left[0,l\right]$ :
$$
\torseurscoh_{S+\rightarrow S-}
=\torseurcol{0}{0}{-F}{0}{0}{0}{A,\left( \vect{x},\vect{y},\vect{z} \right)}
=\torseurcol{0}{0}{-F}{0}{0}{(x-l) F}{G,\left( \vect{x},\vect{y},\vect{z} \right)}
=\torseurcol{N}{T_y}{T_z}{\mathcal{M}_t}{\mathcal{M}_{fy}}{\mathcal{M}_{fz}}{G}
$$
car $\vectm{G}{\text{Ext}}{\text{Poutre}}=\vectm{A}{\text{Ext}}{\text{Poutre}} + \vect{GA}\wedge - F \vect{y}  = (l-x)\vect{x} \wedge -F \vect{y} = (x-l) F \vect{z}$.

\end{corrige}
\else
\fi

\subparagraph{}
\textit{Tracer les diagrammes des sollicitations.}
\ifprof
\begin{corrige}~\\

\begin{center}
\includegraphics[width=.6\linewidth]{images/exo_01_01_corr_02}
\end{center}
\end{corrige}
\else
\fi

\section*{Exercice 3}
\setcounter{subparagraph}{0}
On donne sur le schéma ci-dessous la modélisation d'une poutre et des efforts qui lui sont appliqués.
\begin{center}
\includegraphics[width=.45\textwidth]{images/exo_02}
\end{center}

\subparagraph{}
\textit{Déterminer les actions mécaniques en $A$ et en $B$... et remarquer que cela peut ne servir à rien pour la suite du problème...}
\ifprof
\begin{corrige}
\textit{Remarque :}
Pour déterminer le torseur de cohésion dans ce cas il n'est pas nécessaire de déterminer les actions mécaniques dans la liaison encastrement.

\begin{itemize}[label=\ding{112},font=\color{ocre}] 
\item On isole la poutre.
\item La poutre est soumise à une liaison encastrement en $A$ et à une action mécanique en $C$.
\item On a donc :
\end{itemize}
$$
\torseurcol{X_A}{Y_A}{Z_A}{L_A}{M_A}{N_A}{A}
+\torseurcol{0}{0}{-F}{0}{0}{0}{C}
=\{ 0\}
$$

$\vectm{A}{\text{Ext}}{\text{Poutre}}
=\vectm{C}{\text{Ext}}{\text{Poutre}} + \vect{AC}\wedge -F \vect{z}  
= \left(l\vect{x} - h\vect{z} \right) \wedge -F \vect{z}  
= Fl\vect{y}$

$$
\torseurcol{X_A}{Y_A}{Z_A}{L_A}{M_A}{N_A}{A}
= \torseurcol{0}{0}{F}{0}{-Fl}{0}{A}
$$

\end{corrige}
\else 
\fi


On cherche à déterminer le diagramme des sollicitations dans chacun des tronçons.

\subparagraph{}
\textit{Quels tronçons peut-on considérer ?}
\ifprof
\begin{corrige}~\\
\begin{minipage}[c]{.45\linewidth}
On peut considérer les deux tronçons suivants :
\begin{itemize}[label=\ding{112},font=\color{ocre}] 
\item le tronçon $AB$ sur lequel $x\in[0,l]$;
\item le tronçon $BC$ sur lequel on définit un repère local.
\end{itemize}
\end{minipage} \hfill
\begin{minipage}[c]{.45\linewidth}
\begin{center}
\includegraphics[width=\linewidth]{images/exo_02_corr_01}
\end{center}
\end{minipage}
\end{corrige}
\else 
\fi

\subparagraph{}
\textit{Exprimer le torseur de cohésion dans chacun des tronçons.}

\ifprof
~\\
\begin{corrige} ~\\

\begin{itemize}[label=\ding{112},font=\color{ocre}] 
\item On isole le premier tronçon.
\item Le tronçon est soumis d'une part à l'action mécanique en $A$ (vu qu'on l'a calculée, on va l'utilise ...) et d'autre part à l'action mécanique du torseur de cohésion.
\item On a donc :
\end{itemize}
$$
\torseurscoh_{S-\rightarrow S+} + \torseurcol{0}{0}{-F}{0}{0}{0}{C,\left( \vect{x},\vect{y},\vect{z} \right)} = \{0\}
$$
On a donc :
$$
\torseurscoh_{S+\rightarrow S-}
=\torseurcol{0}{0}{-F}{0}{0}{0}{C,\left( \vect{x},\vect{y},\vect{z} \right)}
$$

$\vectm{G}{\text{Ext}}{\text{Poutre}}
=\vectm{C}{\text{Ext}}{\text{Poutre}} + \vect{GC}\wedge -F \vect{z}  
= \left( (l-x) \vect{x} - h\vect{y} \right) \wedge -F \vect{z}
= F(l-x) \vect{y} +F h\vect{x} 
$
On a donc :
$$
\torseurscoh_{S+\rightarrow S-}
=\torseurcol{0}{0}{-F}{Fh}{F(l-x)}{0}{G,\left( \vect{x},\vect{y},\vect{z} \right)}
$$


\begin{minipage}[c]{.45\linewidth}
\begin{itemize}[label=\ding{112},font=\color{ocre}] 
\item On isole la portion $[GC]$.
\item Le tronçon est soumis d'une part à l'action mécanique en $C$ et d'autre part à l'action mécanique du torseur de cohésion.
\item On a donc :
\end{itemize}
\end{minipage}\hfill
\begin{minipage}[c]{.45\linewidth}
\begin{center}
\includegraphics[width=.8\linewidth]{images/exo_02_corr_02}
\end{center}
\end{minipage}
$$
\torseurscoh_{S-\rightarrow S+} + \torseurcol{0}{0}{-F}{0}{0}{0}{C,\left( \vect{x_1},\vect{y_1},\vect{z} \right)} = \{0\}
$$
On a donc :
$$
\torseurscoh_{S+\rightarrow S-}
=\torseurcol{0}{0}{-F}{0}{0}{0}{C}
$$

$\vectm{G}{\text{Ext}}{\text{Poutre}}
=\vectm{C}{\text{Ext}}{\text{Poutre}} + \vect{GC}\wedge -F \vect{z}  
=  (h-x) \vect{x_1}  \wedge -F \vect{z}
=  (h-x) F \vect{y_1}
$. 
On a donc :
$$
\torseurscoh_{S+\rightarrow S-}
=\torseurcol{0}{0}{-F}{0}{(h-x) F}{0}{G,\left( \vect{x_1},\vect{y_1},\vect{z} \right)}
%=\torseurcol{0}{0}{-F}{(h-x) F}{0}{0}{G,\left( \vect{x},\vect{y},\vect{z} \right)}
$$



\end{corrige}
\else 
\fi

\subparagraph{}
\textit{Tracer les diagrammes des sollicitations.}
\ifprof
\begin{corrige}~\\
\begin{center}
\includegraphics[width=.4\linewidth]{images/exo_02_corr_03}
\end{center}

\end{corrige}
\else 
\fi


\section*{Exercice 4}
\setcounter{subparagraph}{0}
On donne sur le schéma ci-dessous la modélisation d'une poutre et des efforts qui lui sont appliqués.
\begin{center}
\includegraphics[width=.45\textwidth]{images/exo_03}
\end{center}

\subparagraph{}
\textit{Déterminer les actions mécaniques en $A$ et en $B$.}
\ifprof
\begin{corrige}~\\

Le problème est plan. On isole la poutre et on réalise le bilan des actions mécaniques extérieures (figures ci-dessous). 

\begin{minipage}[c]{.47\linewidth}
On applique le théorème de la résultante statique sur $\vect{x}$ puis sur $\vect{y}$ :
\begin{itemize}[label=\ding{112},font=\color{ocre}] 
\item $X_A-F=0$;
\item $Y_A+Y_C-P=0$.
\end{itemize}
On applique le théorème du moment statique en $A$ en projection sur $\vect{z}$ :
\begin{itemize}[label=\ding{112},font=\color{ocre}] 
\item $-\dfrac{Fl}{2}-\dfrac{Pl}{2}+lY_C=0$.
\end{itemize}
On peut alors résoudre le système : 
$$
\left\{
\begin{array}{l}
X_A=F \\
Y_A=-Y_C+P = 0 \\
Y_C = \dfrac{F+P}{2} = F\\
\end{array}
\right.
$$
\end{minipage}\hfill
\begin{minipage}[c]{.47\linewidth}
\begin{center}
\includegraphics[width=.95\linewidth]{images/exo_03_corr_01}
\end{center}
\end{minipage}
\end{corrige}
\else 
\fi


On cherche à déterminer le diagramme des sollicitations dans chacun des tronçons.

\subparagraph{}
\textit{Quels tronçons peut-on considérer ?}
\ifprof
\begin{corrige}
On peut considérer les tronçons $[AB]$, $[BC]$ et $[BD]$.
\end{corrige}
\else 
\fi

\subparagraph{}
\textit{Exprimer le torseur de cohésion dans chacun des tronçons.}
\ifprof
\begin{corrige}
~\\
\begin{itemize}[label=\ding{112},font=\color{ocre}] 
\item On isole la portion $[AG]$ avec $G \in [AC]$.
\item La portion est soumise d'une part à l'action mécanique en $A$, à l'action uniformément répartie (exprimée en $M$, milieu de $[AG]$) et à l'action mécanique du torseur de cohésion.
\item On a donc : 
\end{itemize}
$$
\torseurscoh_{S+\rightarrow S-} +
 \torseurcol{F}{0}{0}{0}{0}{0}{A,\left( \vect{x},\vect{y},\vect{z} \right)} 
 +\torseurcol{0}{-px}{0}{0}{0}{0}{M,\left( \vect{x},\vect{y},\vect{z} \right)} 
 = \{0\}
$$
On a donc, $\forall x \in \left[0,\dfrac{l}{2}\right]$, 
$$
\torseurscoh_{S+\rightarrow S-}
=-\torseurcol{F}{0}{0}{0}{0}{0}{G}
-\torseurcol{0}{-px}{0}{0}{0}{ \dfrac{px^2}{2}}{G}
=\torseurcol{-F}{px}{0}{0}{0}{-\dfrac{px^2}{2}}{G}
$$
car 
$\vectm{G}{\text{Ext}}{\text{Poutre}}
=\vectm{A}{\text{Ext}}{\text{Poutre}} + \vect{GA}\wedge F \vect{x}  
=  -x\vect{x}  \wedge F \vect{x}
=  \vect{0}
$ 

et 
$\vectm{G}{\text{Ext}}{\text{Poutre}}
=\vectm{M}{\text{Ext}}{\text{Poutre}} + \vect{GM}\wedge \left( -px \right) \vect{y}  
= -\dfrac{x}{2}\vect{x}\wedge \left( -px\right) \vect{y}  
=   \dfrac{px^2}{2} \vect{z}  
$.

\begin{itemize}[label=\ding{112},font=\color{ocre}] 
\item On isole la portion $[GC]$ avec $G\in [BC]$.
\item La portion est soumise d'une part à l'action mécanique en $C$, à l'action uniformément répartie (exprimée en $M$, milieu de $[GC]$) et à l'action mécanique du torseur de cohésion en $G$.
\item On a donc : 
\end{itemize}
$$
\torseurscoh_{S-\rightarrow S+} +
 \torseurcol{0}{F}{0}{0}{0}{0}{C,\left( \vect{x},\vect{y},\vect{z} \right)} 
 +\torseurcol{0}{-p(l-x)}{0}{0}{0}{0}{M,\left( \vect{x},\vect{y},\vect{z} \right)} 
 = \{0\}
$$
On a donc, $\forall x \in \left[\dfrac{l}{2},l\right]$, 
$$
\torseurscoh_{S+\rightarrow S-}
= \torseurcol{0}{F}{0}{0}{0}{(l-x)F}{G} 
 +\torseurcol{0}{-p(l-x)}{0}{0}{0}{-p\dfrac{\left(l-x\right)^2}{2}}{G} 
 = \torseurcol{0}{F-p(l-x)}{0}{0}{0}{(l-x)F-p\dfrac{\left(l-x\right)^2}{2}}{G} 
$$
car 
$\vectm{G}{\text{Ext}}{\text{Poutre}}
=\vectm{C}{\text{Ext}}{\text{Poutre}} + \vect{GC}\wedge F \vect{y}  
=  (l-x)\vect{x}  \wedge F \vect{y}
=  (l-x)F\vect{z}
$ 

et 
$\vectm{G}{\text{Ext}}{\text{Poutre}}
=  \vectm{M}{\text{Ext}}{\text{Poutre}} + \vect{GM}\wedge \left( -p(l-x) \right) \vect{y}  
= \left(\dfrac{l-x}{2}\right)\vect{x}\wedge \left(-p(l-x)\right) \vect{y}  
=  -p\dfrac{\left(l-x\right)^2}{2}\vect{z}  
$.

\begin{center}
\includegraphics[width=.45\linewidth]{images/exo_03_corr_02}
\end{center}

\begin{itemize}[label=\ding{112},font=\color{ocre}] 
\item On isole la portion $[GD]$ avec $G\in [BD]$.
\item La portion est soumise d'une part à l'action mécanique en $D$ et à l'action mécanique du torseur de cohésion en $G$.
\item On a donc : 
\end{itemize}
$$
\torseurscoh_{S-\rightarrow S+} +
 \torseurcol{0}{-F}{0}{0}{0}{0}{D,\left( \vect{x_1},\vect{y_1},\vect{z_1} \right)} 
 = \{0\}
 \Leftrightarrow
\torseurscoh_{S+\rightarrow S-} =
 \torseurcol{0}{-F}{0}{0}{0}{F \left(x-\dfrac{l}{2}\right) }{G,\left( \vect{x_1},\vect{y_1},\vect{z_1} \right)}  
$$

car
$\vectm{G}{\text{Ext}}{\text{Poutre}}
=  \vectm{D}{\text{Ext}}{\text{Poutre}} + \vect{GD}\wedge \left(-F \right) \vect{y_1}  
= \left(\dfrac{l}{2}-x\right)\vect{x_1}\wedge \left(-F\right) \vect{y_1}  
=F \left(x-\dfrac{l}{2}\right)  \vect{z_1}  
$.


\end{corrige}
\else 
\fi

\subparagraph{}
\textit{Tracer les diagrammes des sollicitations.}
\ifprof
\begin{corrige}~\\

\begin{center}
\includegraphics[width=.95\linewidth]{images/exo_03_corr_03}
\end{center}
\end{corrige}
\else 
\fi



\section*{Exercice 7}
\setcounter{subparagraph}{0}
On donne sur le schéma ci-dessous la modélisation d'une poutre et des efforts qui lui sont appliqués.
\begin{center}
\includegraphics[width=.45\textwidth]{images/exo_07}
\end{center}



On isole la poutre.

La poutre est soumise à l'action mécanique du mur, ainsi qu'aux efforts en $D$ et en $C$ ;
\begin{itemize}
\item $\vect{M(\vect{F_1},A)} 
= \vect{M(\vect{F_1},D)}+\vect{AD}\wedge\vect{F_1}
= \dfrac{l}{2}\vect{x} \wedge F_1 (-\vect{y})
= -\dfrac{l}{2}F_1  \vect{z} $;
\item $\vect{M(\vect{F_2},A)} 
= \vect{M(\vect{F_2},C)}+\vect{AC}\wedge\vect{F_2}
= \left(l\vect{x} - h\vect{y} \right) \wedge\left(-F_2\vect{z}\right)
= F_2l\vect{y} +F_2h\vect{x} $;
\end{itemize}

On applique le PFS au point $A$ et on a : 
\begin{itemize}
\item $X_A=0$;
\item $Y_A - F_1 = 0$;
\item $Z_A - F_2 = 0$;
\item $L_A +F_2h= 0$;
\item $M_A +F_2l=0 $;
\item $N_A -\dfrac{l}{2}F_1=0 $.
\end{itemize}

\textbf{Tronçon $[AC]$}
\begin{itemize}
\item On isole la partie gauche, soumise à l'action mécanique de l'encastrement et à l'action de la partie $+$ sur la partie $-$.
\item On a donc $$
\torseurscoh_{S+\rightarrow S-} +
 \torseurcol{0}{F_1}{F_2}{-F_2h}{-F_2l}{\dfrac{F_1l}{2}}{A} 
 = \{0\} 
 \Leftrightarrow 
 \torseurscoh_{S+\rightarrow S-} 
 =
 -\torseurcol{0}{F_1}{F_2}{-F_2h}{-F_2l}{\dfrac{F_1l}{2}}{A} 
 =
 -\torseurcol{0}{F_1}{F_2}{-F_2h}{-F_2l+\lambda F_2}{\dfrac{F_1l}{2}-\lambda F_1}{G} 
$$
Avec $\vect{M(G)} 
= \vect{M(A)}+\vect{GA}\wedge\left(F_1 \vect{y}+F_2 \vect{z}\right)
= \vect{M(A)}+\left(- \lambda \vect{x}\right)\wedge\left(F_1 \vect{y}+F_2 \vect{z}\right)
= \vect{M(A)}+\left(- \lambda F_1 \vect{z}\right)+\lambda F_2 \vect{y}
$.
\end{itemize}

\textbf{Tronçon $[CB]$}

\begin{itemize}
\item On isole la partie droite, soumise à l'action mécanique de l'effort en $C$ et à l'action de la partie $-$ sur la partie $+$.
\item On a donc $$
\torseurscoh_{S-\rightarrow S+} +
 \torseurcol{0}{0}{-F_2}{0}{0}{0}{C} 
 = \{0\} 
 \Leftrightarrow 
 \torseurscoh_{S+\rightarrow S-} 
 =
 \torseurcol{0}{0}{-F_2}{0}{0}{0}{C} 
 =
  \torseurcol{0}{0}{-F_2}{F_2 h}{F_2 \lambda}{0}{G} 
$$
Avec $\vect{M(G)} 
= \vect{M(C)}+\vect{GC}\wedge\left(-F_2 \vect{z}\right)
= \left(\lambda \vect{x}-h\vect{y} \right)\wedge\left(-F_2 \vect{z}\right)
= F_2\lambda \vect{y}+F_2 h\vect{x} 
$.
\end{itemize}

\textbf{Tronçon $[BC]$}
\begin{itemize}
\item On se positionne dans le repère local $\left(\vect{x_1},\vect{y_1},\vect{z}\right)$.
\item On isole la partie droite, soumise à l'action mécanique de l'effort en $C$ et à l'action de la partie $-$ sur la partie $+$.
\item On a donc $$
\torseurscoh_{S-\rightarrow S+} +
 \torseurcol{0}{0}{-F_2}{0}{0}{0}{C} 
 = \{0\} 
 \Leftrightarrow 
 \torseurscoh_{S+\rightarrow S-} 
 =
 \torseurcol{0}{0}{-F_2}{0}{0}{0}{C,R_1} 
  = \torseurcol{0}{0}{-F_2}{0}{F_2\lambda }{0}{G,R_1}
    = \torseurcol{0}{0}{-F_2}{F_2\lambda}{0 }{0}{G,R}
$$
Avec $\vect{M(G)} 
= \vect{M(C)}+\vect{GC}\wedge\left(-F_2 \vect{z}\right)
= \left(\lambda \vect{x_1} \right)\wedge\left(-F_2 \vect{z}\right)
= F_2\lambda \vect{y_1}
$.
\end{itemize}

\ifprof
\else
\end{multicols}
\fi
\end{document}


