\documentclass[10pt,fleqn]{article} % Default font size and left-justified equations
\usepackage[%
    pdftitle={RdM : Diagrammes de sollicitations},
    pdfauthor={Xavier Pessoles}]{hyperref}
    
\input{style/new_style}
\input{style/macros_SII}
%\usepackage{unitesi}
\usepackage{multicol}
\usepackage{style/schemabloc}
\usepackage{siunitx}


\fichetrue
%\fichefalse

\proftrue
%\proffalse

\tdtrue
%\tdfalse

%\courstrue
\coursfalse

\def\discipline{Sciences \\Industrielles de \\ l'Ingénieur}
\def\xxtete{Sciences Industrielles de l'Ingénieur}

\def\classe{PT -- PT$\star$}
\def\xxnumpartie{Partie n}
\def\xxpartie{Méthode de résolution permettant la détermination des déformations et des contraintes des pièces déformables\\
Analyse, Modélisation, Résolution}

\def\xxnumchapitre{Chapitre n}
\def\xxchapitre{Titre Chapitre}

\def\xxtitreexo{Exercices d'application -- Détermination du torseur de cohésion.}
\def\xxsourceexo{\hspace{.2cm} }%D'après notes de cours PT -- Lycée G. Eiffel, Bordeaux.}


\def\xxposongletx{2}
\def\xxposonglettext{1.45}
\def\xxposonglety{20}
\def\xxonglet{Part. 2 -- Ch. 2}

\def\xxactivite{Applications}
\def\xxauteur{\textsl{Xavier Pessoles}}

\def\xxcompetences{%
\textsl{%
\textbf{Savoirs et compétences :}\\
\noindent \textbf{Résoudre :} à partir des modèles retenus :
\begin{itemize}[label=\ding{112},font=\color{ocre}] 
\item ***
\item ***
\end{itemize}
\begin{itemize}[label=\ding{112},font=\color{ocre}] 
\item ***
\end{itemize}
%
%\noindent \textit{Mod2 -- C4.1 :} Représentation par schéma bloc.
}}

\def\xxfigures{
\includegraphics[width=.8\textwidth]{images/rdm}
}%figues de la page de garde

\def\xxpied{%
Partie 2 -- Découverte des SLCI -- Analyse, Modélisation, Résolution \\
Ch. 2 : Modélisation des Systèmes Linéaires Continus Invariants -- Transformée de Laplace -- \xxactivite%
}


\setcounter{secnumdepth}{5}
%---------------------------------------------------------------------------


\begin{document}
%\chapterimage{png/Fond_Cin}
\input{style/new_pagegarde}
\vspace{8cm}
\pagestyle{fancy}
\thispagestyle{plain}


\def\columnseprulecolor{\color{ocre}}
\setlength{\columnseprule}{0.4pt} 
\ifprof
\else
\begin{multicols}{2}
\fi

\section*{Exercice 1}
\setcounter{subparagraph}{0}
On considère un ressort de diamètre moyen  $D=\SI{10}{mm}$, de diamètre de fil $d=\si{1}{mm}$, de longueur libre $L_0 = \SI{17,5}{mm}$ de longueur $L_c =\SI{5,55}{mm}$ lorsque les spires sont jointives.  
\includegraphics[width=6cm]{images/ressort_01}

\subparagraph{}\textit{Exprimer le torseur de cohésion en tout point du ressort.}

\ifprof
\begin{corrige}~\\

On isole le tronçon $I$, soumis aux actions du torseur de cohésion et de l'effort en $A$. 
On a donc , 
$$
\torseurscoh_{II \rightarrow I} 
+\left\{\mathcal{F}_{\text{Ext}\rightarrow I}\right\}   = \{0\} 
\Leftrightarrow 
\torseurscoh_{II \rightarrow I} 
= - \left\{\mathcal{F}_{\text{Ext}\rightarrow I}\right\} 
= - \torseurcol{0}{F}{0}{0}{0}{0}{A,\mathcal{R}}
= \torseurcol{ -F \sin \alpha }{-F \cos \alpha }{0}{\dfrac{FD}{2}\cos\alpha}{-\dfrac{FD}{2}\sin\alpha}{0}{G,\mathcal{R}_s}
$$

car :

$$
\vectm{G}{\text{Ext}}{I} 
= \vectm{A}{\text{Ext}}{I} +\vect{GA} \wedge \vectf{\text{Ext}}{I}
= \left(-h \vect{y} +\dfrac{D}{2}\vect{z} \right) \wedge F \vect{y}
= -\dfrac{FD}{2}\vect{x}
= -\dfrac{FD}{2}\left(\cos\alpha \vect{x_s} - \sin\alpha \vect{y_s} \right)
$$

Le ressort est donc soumis à de la traction -- compression, à des effort tranchants, à un moment de torsion et à un moment de flexion.

\end{corrige}
\else
\fi

\subparagraph{}\textit{Déterminer l'expression des contraintes de cisaillement et des contraintes de cisaillement et de torsion. Quelle contrainte semble prédominante ? Justifier.}


\subparagraph{}\textit{Quelle hypothèse permettrait de justifier qu'un ressort n'est soumis qu'à de la torsion.}


\subparagraph{}\textit{Déterminer la flèche d'un tronçon de longueur $\Delta L$. En déduire la flèche totale.}


\subparagraph{}\textit{En déduire la raideur (rigidité ?) du ressort.}
\ifprof
\else
\end{multicols}
\fi
\end{document}


