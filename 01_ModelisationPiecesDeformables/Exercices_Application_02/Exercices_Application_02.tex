\documentclass[10pt,fleqn]{article} % Default font size and left-justified equations
\usepackage[%
    pdftitle={RdM : Diagrammes de sollicitations},
    pdfauthor={Xavier Pessoles}]{hyperref}
    
\input{style/new_style}
\input{style/macros_SII}

\usepackage{multicol}
\usepackage{style/schemabloc}
\fichetrue
%\fichefalse

\proftrue
%\proffalse

\tdtrue
%\tdfalse

%\courstrue
\coursfalse

\def\discipline{Sciences \\Industrielles de \\ l'Ingénieur}
\def\xxtete{Sciences Industrielles de l'Ingénieur}

\def\classe{PT -- PT$\star$}
\def\xxnumpartie{Partie n}
\def\xxpartie{Méthode de résolution permettant la détermination des déformations et des contraintes des pièces déformables\\
Analyse, Modélisation, Résolution}

\def\xxnumchapitre{Chapitre n}
\def\xxchapitre{Titre Chapitre}

\def\xxtitreexo{Exercices d'application -- Détermination du torseur de cohésion.}
\def\xxsourceexo{\hspace{.2cm} D'après notes de cours PT -- Lycée G. Eiffel, Bordeaux.}


\def\xxposongletx{2}
\def\xxposonglettext{1.45}
\def\xxposonglety{20}
\def\xxonglet{Part. 2 -- Ch. 2}

\def\xxactivite{Applications}
\def\xxauteur{\textsl{Xavier Pessoles}}

\def\xxcompetences{%
\textsl{%
\textbf{Savoirs et compétences :}\\
\noindent \textbf{Résoudre :} à partir des modèles retenus :
\begin{itemize}[label=\ding{112},font=\color{ocre}] 
\item ***
\item ***
\end{itemize}
\begin{itemize}[label=\ding{112},font=\color{ocre}] 
\item ***
\end{itemize}
%
%\noindent \textit{Mod2 -- C4.1 :} Représentation par schéma bloc.
}}

\def\xxfigures{
\includegraphics[width=.8\textwidth]{images/rdm}
}%figues de la page de garde

\def\xxpied{%
Partie 2 -- Découverte des SLCI -- Analyse, Modélisation, Résolution \\
Ch. 2 : Modélisation des Systèmes Linéaires Continus Invariants -- Transformée de Laplace -- \xxactivite%
}


\setcounter{secnumdepth}{5}
%---------------------------------------------------------------------------


\begin{document}
%\chapterimage{png/Fond_Cin}
\input{style/new_pagegarde}
\vspace{8cm}
\pagestyle{fancy}
\thispagestyle{plain}


\def\columnseprulecolor{\color{ocre}}
\setlength{\columnseprule}{0.4pt} 
\ifprof
\else
\begin{multicols}{2}
\fi

\section*{Exercice 1}
\setcounter{subparagraph}{0}




\begin{corrige}

En conclusion, 
$$
\torseurscoh_{II \rightarrow I} 
+\left\{\mathcal{T}_{\text{Ext}\rightarrow I}\right\} 
+\left\{\mathcal{T}_{\text{Pression}\rightarrow I} \right\}  = \{0\}
$$

$$
\torseurscoh_{II \rightarrow I}
= 
-\left\{\mathcal{T}_{\text{Ext}\rightarrow I}\right\} 
-\left\{\mathcal{T}_{\text{Pression}\rightarrow I} \right\}  
$$

$$
\torseurscoh_{II \rightarrow I}
= 
-\torseurl{ 
 pR\left( \sin\theta \vect{e_r} + \cos\theta \vect{e_\theta} \right)}{
pR^2 \left( 1- \cos \theta \right)\vect{z}} {P}
-\torseurl{-pR \left(   \sin\theta \vect{e_r}  +\left( \cos\theta -1 \right) \vect{e_\theta}  \right) }{pR^2 \left( \cos\theta -1 \right) \vect{z} }{P} 
$$



$$
\torseurscoh_{II \rightarrow I}
= 
\torseurl{ 
 -pR\left( \sin\theta \vect{e_r} + \cos\theta \vect{e_\theta} \right) +pR \left(   \sin\theta \vect{e_r}  +\left( \cos\theta -1 \right) \vect{e_\theta}  \right) }{
-pR^2 \left( 1- \cos \theta \right)\vect{z} - pR^2 \left( \cos\theta -1 \right) } {P}
$$


$$
\torseurscoh_{II \rightarrow I}
= 
\torseurl{ 
-pR  \vect{e_\theta}  }{
\vect{0} } {P}
$$

\end{corrige}
\else
\fi


\ifprof
\else
\end{multicols}
\fi
\end{document}


