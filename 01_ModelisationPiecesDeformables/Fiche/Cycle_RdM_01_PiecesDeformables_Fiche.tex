\documentclass[10pt,fleqn]{article} % Default font size and left-justified equations
\usepackage[%
    pdftitle={Résistance des matériaux : Modélisation des pièces déformables},
    pdfauthor={Xavier Pessoles}]{hyperref}

\input{style/new_style}
\input{style/macros_SII}

\fichetrue
%\fichefalse

\proftrue
%\proffalse

%\tdtrue
\tdfalse

%\courstrue
\coursfalse

% -------------------------------------
% Déclaration des titres
% -------------------------------------

\def\discipline{Sciences \\Industrielles de \\ l'Ingénieur}
\def\xxtete{Sciences Industrielles de l'Ingénieur}

\def\classe{Fiche PT}
\def\xxnumpartie{Cycle RdM}
\def\xxpartie{Résistance des matériaux}

\def\xxnumchapitre{Chapitre 1}
\def\xxchapitre{\hspace{.12cm} Modélisation des pièces déformables}

\def\xxposongletx{2}
\def\xxposonglettext{1.45}
\def\xxposonglety{19}%16

\def\xxonglet{Cycle RdM -- Ch. 1}

\def\xxactivite{Fiche}
\def\xxauteur{\textsl{Xavier Pessoles}}

\def\xxcompetences{%
\textsl{%
\textbf{Savoirs et compétences :}\\
}}

\def\xxfigures{
%incgraphics[width=.8\textwidth]{}%images/prot_01}
}%figues de la page de garde

\def\xxpied{%
Cycle RdM -- Modélisation des pièces déformables\\
Introduction à la RdM -- \xxactivite%
}

\setcounter{secnumdepth}{5}
%---------------------------------------------------------------------------


\begin{document}
%\chapterimage{png/Fond_Cin}
\input{style/new_pagegarde}
\vspace{2cm}
\pagestyle{fancy}
\thispagestyle{plain}

\section{Hypothèses de la RdM}
\begin{defi}
\textbf{Poutre :} Une poutre d'origine $A$ et et d'extrémité $B$ est un solide engendré par une surface plane $S$ dont une dimension est très grande par rapport aux deux autres. On appelle alors $AB$ la ligne moyenne (ou fibre neutre), $S$ la section droite, perpendiculaire à la ligne moyenne et  $G$ son centre d'inertie. 
\end{defi}

\begin{hypo}
\textbf{Matériaux}
On suppose en RdM que les matériaux sont :
\begin{itemize}
\item \textbf{continus :};
\item \textbf{homogènes :};
\item \textbf{isotropes :};
\item \textbf{élastiques :};
\item \textbf{linéaires:}.
\end{itemize}
\end{hypo}

\begin{center}
\includegraphics[width=.8\linewidth]{images/ElastiqueLineaire}
\end{center}

\begin{hypo}
\textbf{Hypothèse de Navier -- Bernoulli -- Hypothèse cinématique}
\end{hypo}

\begin{hypo}
\textbf{Hypothèse des petits déplacements :} les déplacements petits devront rester petits devant les dimensions de la poutre. 

\end{hypo}

\section{Torseur de cohésion}

\section{Diagramme des sollicitations}

\section{Méthode}




\end{document}


