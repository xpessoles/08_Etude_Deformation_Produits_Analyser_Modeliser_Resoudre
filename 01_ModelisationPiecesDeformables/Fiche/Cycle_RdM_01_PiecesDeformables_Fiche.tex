\documentclass[10pt,fleqn]{article} % Default font size and left-justified equations
\usepackage[%
    pdftitle={Résistance des matériaux : Modélisation des pièces déformables},
    pdfauthor={Xavier Pessoles}]{hyperref}

\input{style/new_style}
\input{style/macros_SII}

\fichetrue
%\fichefalse

\proftrue
%\proffalse

%\tdtrue
\tdfalse

%\courstrue
\coursfalse

% -------------------------------------
% Déclaration des titres
% -------------------------------------

\def\discipline{Sciences \\Industrielles de \\ l'Ingénieur}
\def\xxtete{Sciences Industrielles de l'Ingénieur}

\def\classe{Fiche PT}
\def\xxnumpartie{Cycle 1}
\def\xxpartie{Résistance des matériaux}

\def\xxnumchapitre{Chapitre 1 \vspace{.2cm}}
\def\xxchapitre{\hspace{.12cm} Modélisation des pièces déformables}

\def\xxposongletx{2}
\def\xxposonglettext{1.45}
\def\xxposonglety{19}%16

\def\xxonglet{Cycle RdM -- Ch. 1}

\def\xxactivite{Fiche}
\def\xxauteur{\textsl{Xavier Pessoles}}

\def\xxcompetences{%
\textsl{%
\textbf{Savoirs et compétences :}\\
}}

\def\xxfigures{
%incgraphics[width=.8\textwidth]{}%images/prot_01}
}%figues de la page de garde

\def\xxpied{%
Cycle RdM -- Modélisation des pièces déformables\\
Introduction à la RdM -- \xxactivite%
}

\setcounter{secnumdepth}{5}
%---------------------------------------------------------------------------


\begin{document}
%\chapterimage{png/Fond_Cin}
\input{style/new_pagegarde}
\vspace{2cm}
\pagestyle{fancy}
\thispagestyle{plain}

\section{Hypothèses de la RdM}
\begin{defi}~\\

\noindent\begin{minipage}[c]{.65\linewidth}
\textbf{Poutre :} \\Une poutre d'origine $A$ et et d'extrémité $B$ est un solide engendré par une surface plane $S$ dont une dimension est très grande par rapport aux deux autres. On appelle alors $AB$ la ligne moyenne (ou fibre neutre), $S$ la section droite, perpendiculaire à la ligne moyenne et  $G$ son centre d'inertie. 
\end{minipage} \hfill
\begin{minipage}[c]{.3\linewidth}
\begin{center}
\includegraphics[width=\linewidth]{images/Poutre}
\end{center}
\end{minipage}
\end{defi}

\begin{hypo}
\textbf{Matériaux} \\
On suppose en RdM que les matériaux sont :
\begin{itemize}
\item \textbf{continus :} malgré l'organisation en grains ou en fibres de certains matériaux, on considère que les dimensions de ces grains ou fibres sont négligeables devant les dimensions de la pièce étudie;
\item \textbf{homogènes :} en tous points  les caractéristiques des matériaux sont les mêmes (acier ou plastique, à la différence du béton ou du bois);
\item \textbf{isotropes :} en tous points les caractéristiques mécaniques sont les mêmes dans toutes les directions;
\item \textbf{élastiques :} après suppression des contraintes mécaniques, le matériau retrouve ces dimensions initiales;
\item \textbf{linéaires:} contraintes et déformations sont liées par une loi linéaire.
\end{itemize}
\end{hypo}

\begin{center}
\includegraphics[width=.8\linewidth]{images/ElastiqueLineaire}
\end{center}

\begin{hypo} ~\\

\noindent\begin{minipage}[c]{.65\linewidth}
\textbf{Hypothèse de Navier -- Bernoulli -- Hypothèse cinématique}\\
Lors de la déformation d'une poutre droite, on fait l'hypothèse que le déplacement d'une section droite est un déplacement de corps rigide. Autrement dit, une section plane perpendiculaire à la fibre neutre avant déformation reste perpendiculaire à la fibre neutre après déformation.

(Cette hypothèse n'est plus vérifiée lorsque existe une contrainte de cisaillement.)
\end{minipage} \hfill
\begin{minipage}[c]{.3\linewidth}
\begin{center}
\includegraphics[width=\linewidth]{images/navier}
\end{center}
\end{minipage}
\end{hypo}

\begin{hypo}
\textbf{Hypothèse des petits déplacements :} les déplacements petits devront rester petits devant les dimensions de la poutre. 
\end{hypo}


\section{Torseur de cohésion et sollicitations}

\begin{hypo}
\textbf{Hypothèse de Barré -- Saint Venant :}  ~\\
En s'éloignant suffisant des zones où sont concentrés les efforts, les contraintes et déformations ne dépendent que du torseur de cohésion. 
\end{hypo}


\begin{defi}
\textbf{Torseur de cohésion :} ~\\

Sous une action mécanique extérieure à une poutre, des actions intérieures assurent sa cohésion. Ces actions internes sont modélisées par le torseur de cohésion. 

En subdivisant la poutre en deux tronçons notés $I$ et $II$, puis en isolant la partie $I$, cette dernière est alors soumise aux actions mécaniques de cohésions du tronçon $II$ sur le tronçon $I$ ainsi qu'aux actions mécaniques extérieures. D'après le PFS appliqué à la poutre on a alors :
$\torseurstat{T}{\text{Ext}}{I}+\torseurstat{T}{II}{I} = \left\{0\right\} \Leftrightarrow 
\torseurstat{T}{II}{I} = - \torseurstat{T}{\text{Ext}}{I}$. 

\begin{minipage}[c]{.65\linewidth}

\textbf{Par convention, le torseur de cohésion s'exprime au point $G$ et représente l'action du tronçon $II$ sur le tronçon $I$.} On le notera : $\mathcal{T}_{\text{coh}}$.

\end{minipage} \hfill
\begin{minipage}[c]{.3\linewidth}
\begin{center}
\includegraphics[width=\linewidth]{images/cohesion}
\end{center}
\end{minipage}

\end{defi}



\begin{defi}
\textbf{Sollicitations:} ~\\

\begin{minipage}[c]{.65\linewidth}
On a : 
$$
\{\mathcal{T}_{\text{coh}}\} = \torseurcol{N}{T_y}{T_z}{M_t}{M_{fy}}{M_{fz}}{G,\left(\vx{}, \vy{}, \vz{} \right)}.
$$
\end{minipage} \hfill
\begin{minipage}[c]{.3\linewidth}
\begin{center}
\includegraphics[width=\linewidth]{images/sollicitations}
\end{center}
\end{minipage}

On appelle :
\begin{itemize}
\item $N$ : l'effort normal, induisant un allongement ou un raccourcissement de la poutre;
\item $T_y$ et $T_z$ : les efforts tranchants suivant $\vy{}$ ou $\vz{}$, induisant un glissement des sections;
\item $M_{t}$ : le moment de torsion, induisant un glissement des sections; 
\item $M_{fy}$ et $M_{fz}$ : les moments de flexion autour de $\vy{}$ ou $\vz{}$, induisant une modification de la courbure de la poutre.
\end{itemize}

\end{defi}


\begin{rem}
On peut montrer qu'en un point $G$ d'abscisse $x$:
\begin{itemize}
\item la dérivée de l'effort tranchant est égale à la charge élémentaire appliquée en ce point;
\item la dérivée du moment de flexion autour de $\vect{z}$ est égale à l'opposé de l'effort tranchant $T_y$ en ce point : $\dfrac{\text{d}M_{fz}(x)}{\text{d}x}=-T_y(x)$;
\item la dérivée du moment de flexion autour de $\vect{y}$ est égale à l'effort tranchant $T_z$ en ce point : $\dfrac{\text{d}M_{fy}(x)}{\text{d}x}=T_z(x)$;
\end{itemize}
\end{rem}

\section{Méthode : détermination des diagrammes des sollicitations}
\begin{enumerate}
\item Modélisation du problème : le problème étant donné, il est nécessaire de réaliser un modèle paramétré (schéma cinématique, paramétrage, modélisation des actions mécaniques). 
\item Résolution des actions mécaniques dans les liaisons grâce au PFS. 
\item Découpage de la poutre en tronçons (un nouveau tronçon correspondant à une discontinuité géométrique -- discontinuité de la ligne moyenne -- ou à une discontinuité en effort -- nouvelle charge, répartie ou non, liaison--).
\item Résolution du PFS dans le but de déterminer les torseurs de cohésion.
\item Tracer des diagrammes des sollicitations.
\item Vérifier que $\dfrac{\text{d}M_{fz}(x)}{\text{d}t}=-T_y(x)$.
\end{enumerate}

\begin{center}
\begin{tabular}{|c|c|c|}
\hline
\textbf{Traction} & \textbf{Torsion} & \textbf{Cisaillement} \\
$
\mathcal{T}_{\text{coh}} = \torseurcol{N}{0}{0}{0}{0}{0}{G,\left(\vx{}, \vy{}, \vz{} \right)}.
$
&
$
\mathcal{T}_{\text{coh}} = \torseurcol{0}{0}{0}{M_t}{0}{0}{G,\left(\vx{}, \vy{}, \vz{} \right)}.
$
&
$
\mathcal{T}_{\text{coh}} = \torseurcol{0}{T_y}{T_z}{0}{0}{0}{G,\left(\vx{}, \vy{}, \vz{} \right)}.
$ \\
\hline
\hline
\textbf{Flexion pure} & \textbf{Flexion simple} & \textbf{Flexion + Traction} \\
$
\mathcal{T}_{\text{coh}} = \torseurcol{0}{0}{0}{0}{0}{M_{fz}}{G,\left(\vx{}, \vy{}, \vz{} \right)}.
$
&
$
\mathcal{T}_{\text{coh}} = \torseurcol{0}{T_y}{0}{0}{0}{M_{fz}}{G,\left(\vx{}, \vy{}, \vz{} \right)}.
$
&
$
\mathcal{T}_{\text{coh}} = \torseurcol{N}{T_y}{0}{0}{0}{M_{fz}}{G,\left(\vx{}, \vy{}, \vz{} \right)}.
$ \\
\hline
\hline
\textbf{Flexion + Torsion} & \textbf{Flambage} & \textbf{Flexion déviée} \\
$
\mathcal{T}_{\text{coh}} = \torseurcol{0}{T_y}{0}{M_t}{0}{M_{fz}}{G,\left(\vx{}, \vy{}, \vz{} \right)}.
$
&
$
\mathcal{T}_{\text{coh}} = \torseurcol{N}{0}{0}{0}{0}{M_{fz}}{G,\left(\vx{}, \vy{}, \vz{} \right)}.
$
&
$
\mathcal{T}_{\text{coh}} = \torseurcol{0}{T_y}{T_z}{0}{M_{fy}}{M_{fz}}{G,\left(\vx{}, \vy{}, \vz{} \right)}.
$ \\
\hline
\end{tabular}
\end{center}

\section{Notions de contraintes et déformations}
$$
\mathcal{T}_{\text{coh}} 
= \torseurl{N\vx{}+T_y\vy{}+T_z\vz{}}{M_t\vx{}+M_{fy}\vy{}+M_{fz}\vz{}}{G,\left(\vx{}, \vy{}, \vz{} \right)}
= \torseurl{\int\limits_S \vect{C}\left(M,\vect{x_S}\right) dS}{\int\limits_S \vect{GM} \wedge \vect{C}\left(M,\vect{x_S}\right) dS}{G,\left(\vx{}, \vy{}, \vz{} \right)}
$$
$$
= \torseurl{\int\limits_S \left( \sigma (M) \vect{x_S} + \vect{\tau(M)}  \right) dS}{\int\limits_S \vect{GM} \wedge \vect{C}\left(M,\vect{x_S}\right) dS}{G,\left(\vx{}, \vy{}, \vz{} \right)}
= \torseurl{\int\limits_S \left( \sigma (M) \vect{x_S} + \tau_y \vect{y_S} + \tau_z \vect{z_S}  \right) dS}{\int\limits_S \vect{GM} \wedge \vect{C}\left(M,\vect{x_S}\right) dS}{G,\left(\vx{}, \vy{}, \vz{} \right)}.
$$

\end{document}


