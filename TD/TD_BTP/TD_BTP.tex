\documentclass[10pt,fleqn]{article} % Default font size and left-justified equations
\usepackage[%
    pdftitle={RdM : Diagrammes de sollicitations},
    pdfauthor={Xavier Pessoles}]{hyperref}
    
\input{style/new_style}
\input{style/macros_SII}

\usepackage{multicol}
\usepackage{style/schemabloc}
\fichetrue
%\fichefalse

\proftrue
%\proffalse

\tdtrue
%\tdfalse

%\courstrue
\coursfalse

\def\discipline{Sciences \\Industrielles de \\ l'Ingénieur}
\def\xxtete{Sciences Industrielles de l'Ingénieur}

\def\classe{PT -- PT$\star$}
\def\xxnumpartie{Cycle 2}
\def\xxpartie{Modélisation des sollicitations dans un solide déformable et mesure des déformations.}

\def\xxnumchapitre{}%Chapitre n}
\def\xxchapitre{}%Titre Chapitre}

\def\xxtitreexo{Exercices d'application -- Détermination du torseur de cohésion.}
\def\xxsourceexo{}%\hspace{.2cm} D'après notes de cours PT -- Lycée G. Eiffel, Bordeaux.}


\def\xxposongletx{2}
\def\xxposonglettext{1.45}
\def\xxposonglety{20}
\def\xxonglet{Cycle 2}

\def\xxactivite{Applications}
\def\xxauteur{\textsl{Équipe pédagogique PT -- PT$\star$}}

\def\xxcompetences{%
\textsl{%
\textbf{Savoirs et compétences :}\\
\noindent% \textbf{Résoudre :} à partir des modèles retenus :
\begin{itemize}[label=\ding{112},font=\color{ocre}] 
\item Mod2-C16-S1	: Déterminer le torseur de cohésion dans un solide.
\item Mod2-C16-S2	 : Identifier les sollicitations (traction, compression, flexion, torsion, cisaillement).
\end{itemize}
%
%\noindent \textit{Mod2 -- C4.1 :} Représentation par schéma bloc.
}}

\def\xxfigures{
\includegraphics[width=.8\textwidth]{images/fig_01}
}%figues de la page de garde

\def\xxpied{%
Cycle 2 -- Modélisation des sollicitations \\
%Ch. 2 : Modélisation des Systèmes Linéaires Continus Invariants -- Transformée de Laplace -- 
\xxactivite%
}


\setcounter{secnumdepth}{5}
%---------------------------------------------------------------------------


\begin{document}
%\chapterimage{png/Fond_Cin}
\input{style/new_pagegarde}
\vspace{7cm}
\pagestyle{fancy}
\thispagestyle{plain}


\def\columnseprulecolor{\color{ocre}}
\setlength{\columnseprule}{0.4pt} 
\ifprof
\else
\begin{multicols}{2}
\fi

Sur un hélicoptère, la Boite de Transmission Principale (BTP) permet de distribuer la puissance au rotor principal, au rotor de queue ainsi qu'à différents accessoires (alternateur, pompe hydraulique etc.). Afin d'évaluer la qualité de la BTP, un banc d'essai permet de la solliciter et de recréer les conditions de vol. 

\begin{center}
\begin{tabular}{ccc}
\includegraphics[height=3cm]{images/fig_01} &
\includegraphics[height=3cm]{images/fig_02} &
\includegraphics[height=3cm]{images/fig_03}
\end{tabular}
\end{center}
On étudie la vie ou l'hélicoptère passe d'une condition de vol stationnaire à un déplacement. Cette configuration du banc d'essai se traduit par les efforts suivants en $A$, $B$ et $C$ : 
$$
\left\{\mathcal{T} \left( \text{Ext}_A \rightarrow 1\right) \right\} =
\begin{Bmatrix}
0 & 0 \\
F_y & 0 \\
0 & 0 \\
\end{Bmatrix}_{A,\mathcal{R}_s}
\quad
\left\{\mathcal{T} \left( \text{Ext}_B \rightarrow 1\right) \right\} =
\begin{Bmatrix}
-F_x & 0 \\
-F_y& 0 \\
0 & 0 \\
\end{Bmatrix}_{B,\mathcal{R}_s}
\quad
\left\{\mathcal{T} \left( \text{Ext}_C \rightarrow 2\right) \right\} =
\begin{Bmatrix}
F_x & C_t \\
0 & 0 \\
0 & 0 \\
\end{Bmatrix}_{C,\mathcal{R}_s}.
$$

\subparagraph{}
\textit{Après avoir identifier les différents tronçons à étudier, déterminer le torseur de cohésion dans le solide 1.}

\textbf{On considère le tronçon $[AB]$ pour lequel $\lambda \in ]0,a[$}
\begin{center}
\includegraphics[width=.5\linewidth]{images/corr_01}
\end{center}
\begin{itemize}
\item On isole la partie $II$. 
\item On réalise le bilan des actions mécaniques : 
\begin{itemize}
\item action en $\left\{ A\rightarrow 1 \right\}= \begin{Bmatrix} F_y\overrightarrow{y_s}\\
\overrightarrow{0}\end{Bmatrix}_A = \begin{Bmatrix} F_y\overrightarrow{y_s}\\
 F_A\left( \ell_1 - \lambda \right)\overrightarrow{y_s}\end{Bmatrix}_G$;
\item action en $\left\{ B\rightarrow 1 \right\}= \begin{Bmatrix} F_B\overrightarrow{y_s}\\
\overrightarrow{0}\end{Bmatrix}_A = \begin{Bmatrix} F_A\overrightarrow{y_s}\\
 F_A\left( \ell_1 - \lambda \right)\overrightarrow{y_s}\end{Bmatrix}_G$;
\end{itemize}
\end{itemize}

\subparagraph{Tracer les diagrammes des sollicitations associés à chacune des composantes du torseur de cohésion.}

\ifprof
\else
\end{multicols}
\fi

\end{document}


