\documentclass[10pt,fleqn]{article} % Default font size and left-justified equations
\usepackage[%
    pdftitle={RdM : Diagrammes de sollicitations},
    pdfauthor={Xavier Pessoles}]{hyperref}
    
\input{style/new_style}
\input{style/macros_SII}
\usepackage{siunitx}
\usepackage{multicol}
\usepackage{style/schemabloc}
\fichetrue
%\fichefalse

\proftrue
%\proffalse

\tdtrue
%\tdfalse

%\courstrue
\coursfalse

\def\discipline{Sciences \\Industrielles de \\ l'Ingénieur}
\def\xxtete{Sciences Industrielles de l'Ingénieur}

\def\classe{PT -- PT$\star$}
\def\xxnumpartie{Cycle 2}
\def\xxpartie{Modélisation des sollicitations dans un solide déformable et mesure des déformations.}

\def\xxnumchapitre{}%Chapitre n}
\def\xxchapitre{}%Titre Chapitre}

\def\xxtitreexo{Exercices d'application -- Détermination du torseur de cohésion.}
\def\xxsourceexo{}%\hspace{.2cm} D'après notes de cours PT -- Lycée G. Eiffel, Bordeaux.}


\def\xxposongletx{2}
\def\xxposonglettext{1.45}
\def\xxposonglety{20}
\def\xxonglet{Cycle 2}

\def\xxactivite{Applications}
\def\xxauteur{\textsl{Équipe pédagogique PT -- PT$\star$}}

\def\xxcompetences{%
\textsl{%
\textbf{Savoirs et compétences :}\\
\noindent% \textbf{Résoudre :} à partir des modèles retenus :
\begin{itemize}[label=\ding{112},font=\color{ocre}] 
\item Mod2-C16-S1	: Déterminer le torseur de cohésion dans un solide.
\item Mod2-C16-S2	 : Identifier les sollicitations (traction, compression, flexion, torsion, cisaillement).
\end{itemize}
%
%\noindent \textit{Mod2 -- C4.1 :} Représentation par schéma bloc.
}}

\def\xxfigures{
\includegraphics[width=.8\textwidth]{images/fig_01}
}%figues de la page de garde

\def\xxpied{%
Cycle 2 -- Modélisation des sollicitations \\
%Ch. 2 : Modélisation des Systèmes Linéaires Continus Invariants -- Transformée de Laplace -- 
\xxactivite%
}


\setcounter{secnumdepth}{5}
%---------------------------------------------------------------------------


\begin{document}
%\chapterimage{png/Fond_Cin}
\input{style/new_pagegarde}
\vspace{7cm}
\pagestyle{fancy}
\thispagestyle{plain}


\def\columnseprulecolor{\color{ocre}}
\setlength{\columnseprule}{0.4pt} 
\ifprof
\else
\begin{multicols}{2}
\fi

Sur un hélicoptère, la Boite de Transmission Principale (BTP) permet de distribuer la puissance au rotor principal, au rotor de queue ainsi qu'à différents accessoires (alternateur, pompe hydraulique etc.). Afin d'évaluer la qualité de la BTP, un banc d'essai permet de la solliciter et de recréer les conditions de vol. 

\begin{center}
\begin{tabular}{ccc}
\includegraphics[height=3cm]{images/fig_01} &
\includegraphics[height=3cm]{images/fig_02} &
\includegraphics[height=3cm]{images/fig_03}
\end{tabular}
\end{center}
On étudie la vie ou l'hélicoptère passe d'une condition de vol stationnaire à un déplacement. Cette configuration du banc d'essai se traduit par les efforts suivants en $A$, $B$ et $C$ : 
$$
\left\{\mathcal{T} \left( \text{Ext}_A \rightarrow 1\right) \right\} =
\begin{Bmatrix}
0 & 0 \\
F_y & 0 \\
0 & 0 \\
\end{Bmatrix}_{A,\mathcal{R}_s}
\quad
\left\{\mathcal{T} \left( \text{Ext}_B \rightarrow 1\right) \right\} =
\begin{Bmatrix}
-F_x & 0 \\
-F_y& 0 \\
0 & 0 \\
\end{Bmatrix}_{B,\mathcal{R}_s}
\quad
\left\{\mathcal{T} \left( \text{Ext}_C \rightarrow 2\right) \right\} =
\begin{Bmatrix}
F_x & C_t \\
0 & 0 \\
0 & 0 \\
\end{Bmatrix}_{C,\mathcal{R}_s}.
$$

\subparagraph{}
\textit{Après avoir identifier les différents tronçons à étudier, déterminer le torseur de cohésion dans le solide 1.}

\textbf{On considère le tronçon $[OA]$ pour lequel $\lambda \in ]0,\ell_1[$}
\begin{center}
\includegraphics[width=.5\linewidth]{images/corr_01}
\end{center}
\begin{itemize}
\item On isole la partie $II$. 
\item On réalise le bilan des actions mécaniques : 
\begin{itemize}
\item action en $\left\{ A\rightarrow II \right\}= \begin{Bmatrix} F_y\overrightarrow{y_s}\\
\overrightarrow{0}\end{Bmatrix}_A = \begin{Bmatrix} F_y\overrightarrow{y_s}\\
 F_y\left( \ell_1 - \lambda \right)\overrightarrow{z_s}\end{Bmatrix}_G$;

\item action en $\left\{ B\rightarrow II \right\}= \begin{Bmatrix} -F_x\overrightarrow{x_s}-F_y\overrightarrow{y_s}\\
\overrightarrow{0}\end{Bmatrix}_B= \begin{Bmatrix}-F_x\overrightarrow{x_s}-F_y\overrightarrow{y_s}\\
 -F_y\left( \ell_2 - \lambda \right)\overrightarrow{z_s}\end{Bmatrix}_G$;

\item action en $\left\{ C \rightarrow II \right\}= \begin{Bmatrix} F_x\overrightarrow{x_s}\\
C_t\overrightarrow{x_s}\end{Bmatrix}_B= \begin{Bmatrix} F_x\overrightarrow{x_s}\\
C_t\overrightarrow{x_s}\end{Bmatrix}_G$;
\item $\left\{ \mathcal{T}_{\text{coh}} \left(I \rightarrow II\right) \right\}$
\end{itemize}
\item On applique le PFS à la partie $II$ et on a : 
$$
\left\{ A\rightarrow II \right\} + \left\{ B\rightarrow II \right\} + \left\{ C\rightarrow II \right\}
+\left\{ \mathcal{T}_{\text{coh}} \left(I \rightarrow II\right) \right\} = \{0\}
\Leftrightarrow
\left\{ \mathcal{T}_{\text{coh}} \left(II \rightarrow I\right) \right\}  = \left\{ A\rightarrow II \right\} + \left\{ B\rightarrow II \right\} + \left\{ C\rightarrow II \right\}
$$ 
Ainsi, 
$$
\begin{Bmatrix} 
N & M_t \\
T_y & M_{fy} \\
T_z & M_{fz} 
\end{Bmatrix}_G
=
\begin{Bmatrix} 
0 & C_t \\
0& 0 \\
0 & F_y\left( \ell_1 - \lambda \right) -F_y\left( \ell_2 - \lambda \right) 
\end{Bmatrix}_G
$$
\end{itemize}



\textbf{On considère le tronçon $[AB]$ pour lequel $\lambda \in ]\ell_1,\ell_2[$}
\begin{center}
\includegraphics[width=.5\linewidth]{images/corr_01}
\end{center}
\begin{itemize}
\item On isole la partie $II$. 
\item On réalise le bilan des actions mécaniques : 
\begin{itemize}
\item action en $\left\{ B\rightarrow II \right\}= \begin{Bmatrix} -F_x\overrightarrow{x_s}-F_y\overrightarrow{y_s}\\
\overrightarrow{0}\end{Bmatrix}_B= \begin{Bmatrix}-F_x\overrightarrow{x_s}-F_y\overrightarrow{y_s}\\
 -F_y\left( \ell_2 - \lambda \right)\overrightarrow{z_s}\end{Bmatrix}_G$;

\item action en $\left\{ C \rightarrow II \right\}= \begin{Bmatrix} F_x\overrightarrow{x_s}\\
C_t\overrightarrow{x_s}\end{Bmatrix}_B= \begin{Bmatrix} F_x\overrightarrow{x_s}\\
C_t\overrightarrow{x_s}\end{Bmatrix}_G$;
\item $\left\{ \mathcal{T}_{\text{coh}} \left(I \rightarrow II\right) \right\}$
\end{itemize}
\item On applique le PFS à la partie $II$ et on a : 
$$
\left\{ B\rightarrow II \right\} + \left\{ C\rightarrow II \right\}
+\left\{ \mathcal{T}_{\text{coh}} \left(I \rightarrow II\right) \right\} = \{0\}
\Leftrightarrow
\left\{ \mathcal{T}_{\text{coh}} \left(II \rightarrow I\right) \right\}  = \left\{ B\rightarrow II \right\} + \left\{ C\rightarrow II \right\}
$$ 
Ainsi, 
$$
\begin{Bmatrix} 
N & M_t \\
T_y & M_{fy} \\
T_z & M_{fz} 
\end{Bmatrix}_G
=
\begin{Bmatrix} 
0 & C_t \\
 -F_y& 0 \\
0 &  -F_y\left( \ell_2 - \lambda \right) 
\end{Bmatrix}_G
$$
\end{itemize}


\textbf{On considère le tronçon $[BC]$ pour lequel $\lambda \in ]\ell_2,\ell_3[$}
\begin{center}
\includegraphics[width=.5\linewidth]{images/corr_01}
\end{center}
\begin{itemize}
\item On isole la partie $II$. 
\item On réalise le bilan des actions mécaniques : 
\begin{itemize}
\item action en $\left\{ C \rightarrow II \right\}= \begin{Bmatrix} F_x\overrightarrow{x_s}\\
C_t\overrightarrow{x_s}\end{Bmatrix}_B= \begin{Bmatrix} F_x\overrightarrow{x_s}\\
C_t\overrightarrow{x_s}\end{Bmatrix}_G$;
\item $\left\{ \mathcal{T}_{\text{coh}} \left(I \rightarrow II\right) \right\}$
\end{itemize}
\item On applique le PFS à la partie $II$ et on a : 
$$
\left\{ C\rightarrow II \right\}
+\left\{ \mathcal{T}_{\text{coh}} \left(I \rightarrow II\right) \right\} = \{0\}
\Leftrightarrow
\left\{ \mathcal{T}_{\text{coh}} \left(II \rightarrow I\right) \right\}  =  \left\{ C\rightarrow II \right\}
$$ 
Ainsi, 
$$
\begin{Bmatrix} 
N & M_t \\
T_y & M_{fy} \\
T_z & M_{fz} 
\end{Bmatrix}_G
=
\begin{Bmatrix} 
F_x& C_t \\
0& 0 \\
0 &  0
\end{Bmatrix}_G
$$
\end{itemize}


\subparagraph{}\textit{Tracer les diagrammes des sollicitations associés à chacune des composantes du torseur de cohésion.}
\begin{center}
\includegraphics[width=.95\linewidth]{images/corr_02}
\end{center}

\subparagraph{}\textit{En tenant compte uniquement des contraintes normales, donner la contrainte sous forme vectorielle, déterminer le diamètre minimal de l'arbre dans chacun
des tronçons puis tracer le champ des contraintes dans chacun d'entre eux.}

\begin{itemize}
\item Sur le tronçon $]OA[$ il n'y a aucune contrainte normale. 
\item Sur le tronçon $]AB[$ la contrainte normale due au moment fléchissant s'exprime par  $\overrightarrow{\sigma_2}= -\dfrac{Mfz}{I_{Gz}}y \overrightarrow{x_s}= \dfrac{F_y\left( \ell_2 - \lambda \right)}{I_{Gz}}y \overrightarrow{x_s} $ ($F_y$ étant positif et $\ell_2$ supérieur à $\lambda$ on a la partie supérieure de la poutre en traction et la partie inférieure en compression). Au maximum, la contrainte de traction est donc de $\sigma_{\text{2,max}}=\dfrac{F_y\left( \ell_2 - \ell_1 \right)}{\dfrac{\pi D^4}{64}}\dfrac{D}{2}=32\dfrac{F_y\left( \ell_2 - \ell_1 \right)}{\pi D^3}$. $\sigma_{\text{2,max}} < \dfrac{R_e}{2} \Rightarrow 32\dfrac{F_y\left( \ell_2 - \ell_1 \right)}{\pi D^3} < \dfrac{R_e}{2} \Rightarrow D > \sqrt[3]{\dfrac{64 F_y\left( \ell_2 - \ell_1 \right)}{\pi R_e}}$. \textit{AN: $D= \SI{107}{mm}$.}
\item Sur le tronçon $]BC[$ la contrainte normale est due à la sollicitation de traction : $\overrightarrow{\sigma_{\text{3}}} = \dfrac{N}{S}\overrightarrow{x_s}$. La contrainte étant uniforme dans la section, on a donc 
$\sigma_{\text{3,max}} < \dfrac{R_e}{2} \Leftrightarrow \dfrac{N}{S} < \dfrac{R_e}{2} \Leftrightarrow \dfrac{F_x}{\dfrac{\pi D^2 }{4}} < \dfrac{R_e}{2}
\Leftrightarrow  D > \sqrt{\dfrac{8 F_x}{R_e}} $. \textit{AN: $D= \SI{33}{mm}$.}
\end{itemize}

\begin{center}
\includegraphics[width=.5\linewidth]{images/corr_03}
\end{center} 

\subparagraph{}\textit{Faire de même en ne tenant compte que des contraintes de cisaillement.}



\begin{itemize}
\item L'ensemble de la poutre étant en torsion, on a : $ \overrightarrow{\tau_t}=\dfrac{M_t}{I_{0}}\rho \overrightarrow{e_{\theta}}.$ La contrainte est maximale lorsque le rayon est maximal. Ainsi, $ \tau_{t,\text{max}} = \dfrac{M_t}{I_{0}}\dfrac{D}{2} <\dfrac{R_e}{2}\dfrac{1}{2} \Leftrightarrow \dfrac{M_t}{\dfrac{\pi D^4}{32}}\dfrac{D}{2} <\dfrac{R_e}{4} \Leftrightarrow D > \sqrt[3]{\dfrac{64 M_t}{\pi R_e}} $. 
\textit{AN: $D= \SI{52}{mm}$.}
\item Le tronçon $]AB[$ étant soumis à de la flexion simple, on a donc du cisaillement : $\overrightarrow{\tau_{2}} = \dfrac{T_y}{S}\overrightarrow{y_s}$. 
L'effort tranchant étant uniforme, on a $\tau_{\text{max},2} < \dfrac{R_e}{2}\dfrac{1}{2} \Leftrightarrow  \dfrac{F_y}{\dfrac{\pi D^2}{4}}< \dfrac{R_e}{4} \Leftrightarrow D > \sqrt{\dfrac{8 F_y}{R_e \pi }}$. \textit{AN: $D= \SI{23}{mm}$.}
\end{itemize}
\ifprof
\else
\end{multicols}
\fi


\begin{center}
\includegraphics[width=.5\linewidth]{images/corr_04}
\end{center}

\subparagraph{}\textit{N'étant pas possible de sommer des contraintes normales et des contraintes de cisaillement, on peut utiliser le critère de von Mises : $\sigma_{\text{vm}}=\sqrt{\sigma_{\text{max}}^2+3\tau_{\text{max}}^2}\leq R_e$. Exprimer $\sigma_{\text{vm}}$ en fonction de $\rho$ et $\theta$. La courbe suivante représente la fonction $\dfrac{R_e}{2}-\sigma_{\text{vm}}$ pour $\theta$ compris entre et $\left[-\pi,\pi\right]$ pour différents diamètres. Proposer un diamètre de l'arbre pour le tronçon $]AB[$.}

Calcul de la contrainte de torsion : 
$\overrightarrow{\tau}=\dfrac{M_t}{I_{0}}\rho \overrightarrow{e_{\theta}} - \dfrac{F_y}{\pi \rho^2} \vect{y_s}
= \dfrac{M_t}{I_{0}}\rho \left( \cos\theta \overrightarrow{y_{s}}-\sin\theta \overrightarrow{z_{s}}\right) - \dfrac{F_y}{\pi \rho^2} \vect{y_s}$. 
On a alors $||\overrightarrow{\tau}||^2 = \left(\dfrac{M_t}{I_{0}}\rho  \cos\theta -  \dfrac{F_y}{\pi \rho^2} \right)^2 +   \left(\dfrac{M_t}{I_{0}}\rho  \sin\theta  \right)^2$. 
En conséquences, 
$\sigma_{\text{vm}}^2\left( \rho,\theta\right)=\left( \dfrac{4 F_y\left( \ell_2 - \ell_1 \right)}{\pi \rho^3} \right)^2 + 3 \left(\dfrac{M_t}{I_{0}}\rho  \cos\theta -  \dfrac{F_y}{\pi \rho^2} \right)^2 +   \left(\dfrac{M_t}{I_{0}}\rho  \sin\theta  \right)^2$.

On cherche donc $\rho$ tel que $\dfrac{R_e}{2}-\sigma_{\text{vm}}>0$.

%Sur le tronçon $]AB[$, pour un point appartenant à l'extérieur de l'arbre, situé sur l'axe $\left(O,\overrightarrow{y_s}\right)$, les contraintes normales et tangentielles sont maximales. En tenant compte du coefficient de sécurité, on a donc :
%$\sqrt{\sigma_{\text{max}}^2+3\tau_{\text{max}}^2}\leq \dfrac{R_e}{2} 
%\leftrightarrow
%\left( \dfrac{32F_y\left( \ell_2 - \ell_1 \right)}{\pi D^3} \right)^2+3\left( \right)^2\leq \dfrac{R_e^2}{4} 
%$



\begin{center}
\includegraphics[width=.5\linewidth]{images/untitled}
\end{center}

\subparagraph{}\textit{En conservant les dimensions précédentes et en considérant une phase de vie où l'arbre n'est soumis que de la torsion, déterminer le rayon de raccordement pour que le coefficient de concentration de contrainte soit inférieur à 1,5.}

\begin{center}
\includegraphics[width=.5\linewidth]{images/concentration_contrainte_torsion}
\end{center}


\subparagraph{}\textit{Proposer une géométrie de l'arbre.}
\end{document}


