\documentclass[10pt,fleqn]{article} % Default font size and left-justified equations
\usepackage[%
    pdftitle={Résistance des matériaux : Modélisation des pièces déformables - Torsion},
    pdfauthor={Xavier Pessoles}]{hyperref}

\input{style/new_style}
\input{style/macros_SII}

\fichetrue
%\fichefalse

\proftrue
%\proffalse

%\tdtrue
\tdfalse

%\courstrue
\coursfalse

% -------------------------------------
% Déclaration des titres
% -------------------------------------

\def\discipline{Sciences \\Industrielles de \\ l'Ingénieur}
\def\xxtete{Sciences Industrielles de l'Ingénieur}

\def\classe{Fiche PT}
\def\xxnumpartie{Cycle RdM}
\def\xxpartie{Résistance des matériaux}

\def\xxnumchapitre{Chapitre 3 \vspace{.2cm}}
\def\xxchapitre{\hspace{.12cm} Modélisation des pièces déformables en torsion}

\def\xxposongletx{2}
\def\xxposonglettext{1.45}
\def\xxposonglety{19}%16

\def\xxonglet{Cycle RdM -- Ch. 3}

\def\xxactivite{Fiche}
\def\xxauteur{\textsl{Xavier Pessoles}}

\def\xxcompetences{%
\textsl{%
\textbf{Savoirs et compétences :}\\
}}

\def\xxfigures{
%incgraphics[width=.8\textwidth]{}%images/prot_01}
}%figues de la page de garde

\def\xxpied{%
Cycle RdM -- Modélisation des pièces déformables\\
Torsion -- \xxactivite%
}

\setcounter{secnumdepth}{5}
%---------------------------------------------------------------------------


\begin{document}
%\chapterimage{png/Fond_Cin}
\input{style/new_pagegarde}
\vspace{2cm}
\pagestyle{fancy}
\thispagestyle{plain}

\section{Définitions}
\begin{defi} -- \textbf{Torseur des sollicitations}  ~\\

\begin{minipage}[c]{.65\linewidth}
Pour une sollicitation en torsion, le torseur de cohésion est de la forme : 
$$
\{\mathcal{T}_{\text{coh}}\} = \torseurcol{0}{0}{0}{M_t}{0}{0}{G,\left(\vx{}, \vy{}, \vz{} \right)}.$$
\end{minipage} \hfill
\begin{minipage}[c]{.3\linewidth}
\begin{center}
\includegraphics[width=\linewidth]{images/torsion}
\end{center}
\end{minipage}



\end{defi}

\begin{defi} -- 
\textbf{Contrainte et déformations longitudinales} ~\\
\begin{minipage}[c]{.7\linewidth}
En torsion, la contrainte est uniquement tangentielle et dépend de la position radiale dans la section: 
$$ \tau = \dfrac{M_t}{I_{0}}r
\quad 
\text{avec}
\begin{array}{ll}
\tau &\text{contrainte tangentielle en \text{MPa},}\\
M_t &\text{moment de torsion en Nm,} \\
I_{0} &\text{moment quadratique polaire en mm}^4 \text{ et} \\
r &\text{rayon en mm.}
\end{array}
$$
L'angle unitaire de torsion est défini par 
$$
\theta(x) = \dfrac{\text{d} \varphi}{\text{d} x} 
\quad 
\text{avec}
\begin{array}{ll}
\theta &\text{angle unitaire de torsion en rad/m,}\\
\text{d} \varphi &\text{} \\
\text{d} x &\text{.} \\
\end{array}
$$

La déformation est définie par : 
$$
\gamma = r \dfrac{\text{d} \varphi}{\text{d} x} = r\theta  
\quad 
\text{avec}
\begin{array}{ll}
\theta &\text{angle unitaire de torsion en rad/m,}\\
\text{d} \varphi &\text{} \\
\text{d} x &\text{.} \\
\end{array}
$$
\end{minipage} \hfill
\begin{minipage}[c]{.25\linewidth}
\begin{center}
\includegraphics[width=.9\linewidth]{images/contrainte_torsion}
\end{center}
\end{minipage}
\end{defi}


\begin{resultat}
\textbf{Loi de comportement -- Loi de Hooke} ~\\
Lorsqu'un matériau est sollicité dans un domaine élastique, contrainte et déformation sont liées par la loi de Hooke : 
$$ \tau = G \gamma
\quad 
\text{avec}
\begin{array}{ll}
\sigma &\text{contrainte en \text{MPa}},\\
\varepsilon &\text{déformation sans dimension,}\\
E &\text{module de Young en MPa (N}\cdot\text{mm}^2\text{).}
\end{array}
$$
\end{resultat}

\end{document}


